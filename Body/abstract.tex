% $Log: abstract.tex,v $
% Revision 1.1  93/05/14  14:56:25  starflt
% Initial revision
%
% Revision 1.1  90/05/04  10:41:01  lwvanels
% Initial revision
%
%
%% The text of your abstract and nothing else (other than comments) goes here.
%% It will be single-spaced and the rest of the text that is supposed to go on
%% the abstract page will be generated by the abstractpage environment.  This
%% file should be \input (not \include 'd) from cover.tex.

In large programming classes, a single problem may yield thousands of student solutions. Solutions can vary in correctness, approach, and readability. Understanding large-scale variation in solutions is a hard but important problem. For teachers, this variation could be a source of innovative new student solutions and instructive examples. Understanding solution variation could help teachers write better feedback, test cases, and evaluation rubrics. Theories of learning, e.g., analogical learning and variation theory, suggest that students would benefit from understanding the variation in their peers' solutions as well. Even when there are many solutions to a problem, when a student is struggling in a large class, other students may have struggled along a similar solution path, hit the same bugs, and have hints based on that earned expertise.

This thesis describes three systems that exploit the value of solution variation in large-scale programming classes. All three systems have been evaluated using data or live deployments in on-campus or edX courses with thousands of students. (1) OverCode visualizes thousands of programming solutions using static and dynamic analysis to cluster similar solutions. Compared to the status quo tools, OverCode lets teachers quickly develop a high-level view of student understanding and misconceptions and provide feedback that is relevant to more student solutions. (2) Foobaz helps teachers give feedback at scale on a critical aspect of readability, i.e., variable naming. Foobaz displays the distribution of student-chosen names for each common variable in student solutions and, with a few teacher annotations, it generates personalized quizzes that help students learn from the good and bad naming choices of their peers. (3) ClassOverflow collects and organizes hints indexed by the autograder test that failed or a performance characteristic like size or speed. It helps students reflect on their debugging or optimization process and generates hints that can help other students with the same problem. These systems demonstrate how clustering and visualizing solution variation can help teachers directly respond to trends and outliers within student solutions, as well as help students help each other. %The feedback generated by both teachers and students can be reused by future students.%, and could potentially bootstrap an intelligent tutor tailored to the problem.

