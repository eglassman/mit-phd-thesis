

\section{OverCode}

There is a growing body of work on both the frontend and backend required to manage and present the large volumes of solutions gathered from MOOCs, intelligent tutors, online learning platforms, and large residential classes. The backend necessary to analyze solutions expressed as code has followed from prior work in fields such as program analysis, compilers, and machine learning. A common goal of this prior work is to help teachers monitor the state of their class, or provide solution-specific feedback to many students. However, there has not been much work on developing interactive user interfaces that enable a teacher to navigate the large space of student solutions. 

We first present here a brief review of the state of the art in the backend, specifically about analyzing code generated by students who are independently attempting to implement the same function. This will place our own backend in context. We then review the information visualization principles and systems that inspired our frontend contributions.

\subsection{Related Work in Program Analysis}

\todo{cite AspiringMinds work "Question Independent Grading using Machine Learning: The Case of Computer Program Grading"}

\subsubsection{Canonicalization and Semantics-Preserving Transformations}

When two pieces of code have different syntax, and therefore different abstract syntax trees (ASTs), they may still be semantically equivalent. A teacher viewing the code may want to see those syntactic differences, or may want to ignore them in order to focus on semantic differences. Semantics-preserving transformations can reduce or eliminate the syntactic differences between code. Applying semantics-preserving transformations, sometimes referred to as canonicalization or standardization, has been used for a variety of applications, including detecting clones \cite{baxter} and automatic ``transform-based diagnosis’’ of bugs in students’ programs written in programming tutors \cite{xutransformation}. 

OverCode also canonicalizes solutions, using variable renaming. OverCode’s canonicalization is novel in that its design decisions were made to maximize {\it human readability} of the resulting code. As a side-effect, syntactic differences between answers are also reduced.

\subsubsection{Abstract Syntax Tree-based Approaches}

Huang et al. \citeyear{MOOCshop} worked with short Matlab/Octave functions submitted online by students enrolled in a machine learning MOOC. The authors generate an AST for each solution to a problem, and calculate the tree edit distance between all pairs of ASTs, using the dynamic programming edit distance algorithm presented by Shasha et al. \citeyear{shasha1994exact}. Based on these computed edit distances, clusters of syntactically similar solutions are formed. The algorithm is quadratic in both the number of solutions and the size of the ASTs. Using a computing cluster, the Shasha algorithm was applied to just over a million solutions. 

Calculating tree-edit distances between all pairs of ASTs allows Huang et al. to analyze differences within each line. It’s also computationally expensive, with quadratic complexity both in the number of solutions and the size of the ASTs~\cite{MOOCshop}. The OverCode analysis pipeline does not reason about differences any finer than a line of code, but it has linear complexity in the number of solutions and in the size of the ASTs.

Codewebs \cite{codewebs} created an index of ``code phrases'' for over a million submissions from the same MOOC and semi-automatically identified equivalence classes across these phrases, using a data-driven, probabilistic approach. The Codewebs search engine accepts queries in the form of subtrees, subforests, and contexts that are subgraphs of an AST. A teacher labels a set of AST subtrees considered semantically meaningful, and then queries the search engine to extract all equivalent subtrees from the dataset. OverCode does analyze the AST of student solutions but only in order to reformat code and rename variables that behave similarly on a test case. All further code comparison is done through string matching lines of code that have consistent formatting and variable names.

Both Codewebs \cite{codewebs} and Huang et al. \citeyear{MOOCshop} use unit test results and AST edit distance to identify clusters of submissions that could potentially receive the same feedback from a teacher. These are non-interactive systems that require hand-labeling in the case of Codewebs, or a computing cluster in the case of Huang et al. In contrast, OverCode’s pipeline does not require hand-labeling and runs in minutes on a laptop, then presents the results in an interactive user interface.

\subsubsection{Supervised Machine Learning and Hierarchical Pairwise Comparison}

Semantic equivalence is another way of saying that two solutions have the same schema. A {\em schema}, in the context of programming, is a high-level cognitive construct by which humans understand or generate code to solve problems \cite{Soloway1984}. For example, two programs that implement bubble sort have the same schema, bubble sort, even though they may have different low-level implementations. Taherkhani et al. \citeyear{taherkhani12,taherkhani13} used supervised machine learning methods to successfully identify which of several sorting algorithms a solution used. Each solution is represented by statistics about language constructs, measures of complexity, and detected roles of variables. Variable roles are determined based on variable behavior. OverCode identifies common variables based on variable behavior as well. Both methods consider the sequence of values that variables are assigned to, but OverCode does not attempt to categorize variable behavior as one of a set of predefined roles. Similarly, Taherkhani et al.’s method can identify sorting algorithms that have already been analyzed and included in its training dataset. OverCode, in contrast, handles problems for which the algorithmic schema is not already known. 

Luxton-Reilly et al. \citeyear{Luxton13} label types of variations as structural, syntactic, or presentation-related. The structural similarity between solutions in a dataset is captured by comparing their control flow graphs. If the control flow of two solutions is the same, then the syntactic variation within the blocks of code is compared by looking at the sequence of token classes. Presentation-based variation, such as variable names and spacing, is only examined when two solutions are structurally and syntactically the same. In contrast, our approach is not hierarchical, and uses dynamic information in addition to syntactic information.

\subsubsection{Program Synthesis}

There has also been work on analyzing each student solution individually to provide more precise feedback. Singh et al. \citeyear{rishabh} use a constraint-based synthesis algorithm to find the minimal changes needed to make an incorrect solution functionally equivalent to a reference implementation. The changes are specified in terms of a problem-specific error model that captures the common mistakes students make on a particular problem.

Rivers and Koedinger \citeyear{riversaied} propose a data-driven approach to create a solution space consisting of all possible paths from the problem statement to a correct solution. To project code onto this solution space, the authors apply a set of normalizing program transformations to simplify, anonymize, and order the program’s syntax. The solution space can then be used to locate the potential learning progression for a student submission and provide hints on how to correct their attempt. Unlike OverCode’s variable renaming method, which reflects the most common names chosen by students, Rivers and Koedinger replace student variable names with arbitrary symbols, i.e. \codevar{daysInMonth} might be mapped to \codevar{v0}. 

Singh et al. and Rivers and Koedinger focus on providing hints to students along their path to a correct solution. Instead of providing hints, the aim of our work is to help instructors navigate the space of \emph{correct} solutions and therefore techniques based on checking only the functional correctness are not helpful in computing similarities and differences between such solutions.

\subsubsection{Code Comparison Tools}
File comparison tools, such as Apple FileMerge, Microsoft WinDiff, and Unix diff, are a class of tools that analyze and present differences between files. Highlighting indicates inserted, deleted, and changed text. Unchanged text is collapsed. Some of these tools are customized for analyzing code, such as Code Compare. They are also integrated into existing integrated development environments (IDE), including IntelliJ IDEA and Eclipse. These code-specific comparison tools may match methods rather than just comparing lines. Three panes side-by-side are used to show code during three-way merges of file differences. There are tools, e.g. KDiff3, which will show the differences between four files when performing a distributed version control merge operation, but that appears to be an upper limit. These tools do not scale beyond comparing a handful of programs simultaneously. OverCode can show hundreds or thousands of solutions simultaneously, and its visualization technique dims the lines that are shared with the most common solution, rather than using colors to indicate inserted or deleted lines.

MOSS~\cite{schleimer2003winnowing} is a widely used system for finding similarities across student solutions for detecting plagiarism. MOSS uses a windowing technique to select fingerprints from hashes of $k$-grams from a solution. It first creates an index mapping fingerprints to corresponding locations for all solutions. It then fingerprints each solution again to compute the list of matching fingerprints for the solution. Finally, it rank-orders the fingerprint matches by their size for each pair of solution match. This algorithm enables MOSS to find partial matches between two solutions that are in different positions with good accuracy. OverCode, on the other hand, uses a simple linear algorithm to create stacks of solutions with the same canonical form. It uses an equivalence based on the set of statements in a solution to capture position-independent statement matches.

\subsection{Related Work in User Interfaces for Solution Visualization}

Several user interfaces have been designed for providing grades or feedback to students at scale, and for browsing large collections in general, not just student solutions. 

Basu et al. \citeyear{basupowergrading} provide a novel user interface for {\it powergrading} short-answer questions. Powergrading means assigning grades or writing feedback to many similar answers at once. The backend uses machine learning that is trained to cluster answers, and the frontend allows teachers to read, grade or provide feedback to those groups of similar answers simultaneously. Teachers can also discover common misunderstandings. The value of the interface was verified in a study of 25 teachers looking at their visual interface with clustered answers. When compared against a baseline interface, the teachers assigned grades to students substantially faster, gave more feedback to students, and developed a ``high-level view of students’ understanding and misconceptions’’ \cite{basuDivideAndConquer}.

%Beyond powergrading, there is a variety of related work in the field of information visualization, which is focused on the visual presentation of data to aid human cognition. Flamenco\cite{flamenco} was a faceted browsing interface for a large collection of art. made use of faceted metadata, and was designed for exploring a large collection, which was a difficult task using traditional query-based interfaces. The interface was tested by 32 art history students to browse 35,000 images of art. OverCode is also about making a large collection easily browsable are comparable to Flamenco, though our collection is code submitted by students, and our metadata is produced by our program analysis pipeline.

At the intersection of information visualization and program analysis is Cody\footnote{\url{mathworks.com/matlabcentral/cody}}, an informal learning environment for the Matlab programming language. Cody does not have a teaching staff but does have a {\em solution map} visualization to help students discover alternative ways to solve a problem. A solution map plots each solution as a point against two axes: time of submission on the horizontal axis, and code size on the vertical axis, where \textit{code size} is the number of nodes in the parse tree of the solution. Despite the simplicity of this metric, solution maps can provide quick and valuable insight when assessing large numbers of solutions~\cite{ICERGlassman}.

% Originally launched in January of 2012, there are over 1500 problems posted by and for users, and users have submitted over 281,000 solutions. 
%  (rcm: took this out because it’s not obvious that this has anything to do with the solution map
% Though the information visualizations continue to be updated on the Cody website, screenshots from the site during the Summer of 2013 have been published \ref{ICERGlassman}.
%  (rcm: took this out because it’s not relevant -- people can go look at the Cody website, right?)

OverCode has also been inspired by information visualization projects like WordSeer \cite{wordseerlitcomp13,wordseercikm13} and CrowdScape \cite{crowdscape}. WordSeer helps literary analysts navigate and explore texts, using query words and phrases \cite{wordseerhcir11}. CrowdScape gives users an overview of crowd-workers’ performance on tasks. An overview of crowd-workers each performing on a task, and an overview of submitted code, each executing a test case, are not so different, from an information presentation point of view.