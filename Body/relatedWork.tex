%% This is an example first chapter.  You should put chapter/appendix that you
%% write into a separate file, and add a line \include{yourfilename} to
%% main.tex, where `yourfilename.tex' is the name of the chapter/appendix file.
%% You can process specific files by typing their names in at the 
%% \files=
%% prompt when you run the file main.tex through LaTeX.
\chapter{Related Work}\label{chapter:relatedwork}

Systems that help students in massive programming courses may build on work from any or all the following related fields: program analysis, program synthesis, crowd workflows, user-interface design, machine learning, and learning science. First, I present prior work and theories of how people learn that later inspired key design decisions. I then clarify how this thesis work is novel by describing related work that achieves similar goals or uses similar methods.

Computers' potential as teaching aids was recognized soon after their development; not long after it was physically possible to bring a computer into the classroom, they were used as vehicles for education \cite{computersInEdu}. Modern computer-aided instruction includes intelligent tutoring systems, automated tutorials, power-grading systems, and massive open online course platforms. 

\todo{add cody! (see ICER submission)}
\todo{look into PHOG and other interesting work here: http://www.srl.inf.ethz.ch/spas.php http://www.srl.inf.ethz.ch/raychev.php and "Predicting Program Properties from “Big Code” http://www.srl.inf.ethz.ch/papers/jsnice15.pdf and "Code Completion with Statistical Language Models" http://www.srl.inf.ethz.ch/papers/pldi14-statistical.pdf}

\todo{Analyzing Engineering Design through the Lens of Computation
Authors
Marcelo Worsley, Paulo Blikstein}

\todo{Teaching composition quality at scale: human judgment in the age of autograders
Authors
John DeNero, Stephen Martinis}

\todo{Problems Before Solutions: Automated Problem Clarification at Scale
Authors
Soumya Basu, Albert Wu, Brian Hou, John DeNero}

\todo{CS10K Teachers by 2017?: Try CS1K+ students NOW! Coping with the Largest CS1 Courses in History
Authors
Daniel D Garcia, Jennifer Campbell, John DeNero, Mary Lou Dorf, Stuart Reges}

\todo{Fuzz Testing Projects in Massive Courses
Authors
Sumukh Sridhara, Brian Hou, Jeffrey Lu, John DeNero}

\todo{https://computinged.wordpress.com/2012/12/14/research-questions-on-moocs-whos-talking-whos-completing-and-wheres-the-teaching/}

\todo{https://computinged.wordpress.com/2012/08/14/daphne-kollers-ted-talk-whats-new-about-moocs/

The Relative Effectiveness of Human Tutoring,
Intelligent Tutoring Systems, and Other Tutoring
Systems
KURT VanLEHN $http://www.public.asu.edu/~kvanlehn/Stringent/PDF/EffectivenessOfTutoring_Vanlehn.pdf$}

\todo{cite Techniques for Plan Recognition
SANDRA CARBERRY for solution path mining, but I did not do that...}
\todo{Automated Feedback Generation for

Introductory Programming Assignments

Rishabh Singh}

\todo{Learning Design Patterns
with Bayesian Grammar Induction
Jerry O. Talton et al. http://graphics.stanford.edu/~lfyg/gi.pdf}

\todo{http://cs.stanford.edu/people/sharmar/pubs/ddec.pdf Data-driven equivalence checking}

\todo{Mining Source Code Repositories at Massive Scale
using Language Modeling
Miltiadis Allamanis, Charles Sutton}

\todo{Student coding styles as predictors of help-seeking behavior
Authors
Engin Bumbacher, Alfredo Sandes, Amit Deutsch, Paulo Blikstein}

\todo{see library on Google Scholar, Vineet's review of literature}

\todo{Programming Pathways: A Technique for Analyzing Novice Programmers’ Learning Trajectories
Authors
Marcelo Worsley, Paulo Blikstein}

\todo{Educational data mining and learning analytics: Applications to constructionist research
Authors
Matthew Berland, Ryan S Baker, Paulo Blikstein}

\todo{cite WebZeitGeist}

\todo{Programming pluralism: Using learning analytics to detect patterns in the learning of computer programming
Authors
Paulo Blikstein, Marcelo Worsley, Chris Piech, Mehran Sahami, Steven Cooper, Daphne Koller}

\todo{cite Using learning analytics to assess students' behavior in open-ended programming tasks
Authors
Paulo Blikstein}

\todo{Modeling how students learn to program
Authors
Chris Piech, Mehran Sahami, Daphne Koller, Steve Cooper, Paulo Blikstein}

\todo{see Related Work folder in Google Drive and Question Independent Grading using Machine Learning:

The Case of Computer Program Grading

Gursimran Singh

Shashank Srikant

Varun Aggarwal}

\todo{add The Sweep: Essential Examples for In-Flow Peer Review by
JG Politz, JM Collard, A Guha, K Fisler, S Krishnamurthi and In-flow peer-review of tests in test-first programming
Authors
Joe Gibbs Politz, Shriram Krishnamurthi, Kathi Fisler and In-Flow Peer Review
Authors
Dave Clarke, Tony Clear, Kathi Fisler, Matthias Hauswirth, Shriram Krishnamurthi, Joe Gibbs Politz, Ville Tirronen, Tobias Wrigstad and CaptainTeach: a platform for in-flow peer review of programming assignments
Authors
Joe Gibbs Politz, Shriram Krishnamurthi, Kathi Fisler and CaptainTeach: Multi-stage, in-flow peer review for programming assignments
Authors
Joe Gibbs Politz, Daniel Patterson, Shriram Krishnamurthi, Kathi Fisler}

\todo{add this to related work https://computinged.wordpress.com/2016/05/16/implementing-design-studio-pedagogy-with-an-augmented-reality-cs-classroom/}

\section{Learning from Variation}

\begin{comment}
Marton et al.'s variation theory \cite{Marton13} holds that in order to learn something, one must see examples that vary along particular dimensions: ``contrast,'' as in pairing it with something it is not; ``generalization,'' as in presenting multiple instances of the object or concept to be learned, varying only that which is irrelevant; ``separation,'' as in presenting multiple instances of the object or concept, varying only that which can vary internally without changing the object or concept into something else; and ``fusion,'' as in seeing multiple examples in which previously analytically separated aspects must be processed together to recognize the object or concept. The aspects which are related to these dimensions of variation and therefore define the object or concept are called ``critical features.''
\end{comment}

\subsection{Variation Theory}
Marton's Variation Theory, as summarized by Suhonen et al. \cite{suhonen}, is defined by the dimensions of variation necessary to fully communicate a concept to a student: \emph{contrast} (``in order to experience something, a person must experience something else to compare it with''); \emph{generalization}, or the ways something can vary without becoming something else; \emph{separation}, or looking at the variation only across specific features; and \emph{fusion}, where multiple critical aspects of the concept are varied simultaneously. In other words, variation reveals which aspects of a phenomenon are superficial/irrelevant and which are innate/critical to its definition \cite{Leung}. These aspects that define the object or concept are called ``critical features.'' The Variation Theory is a framework that now guides the design of some critical reading exercises \cite{Tong} and exercises for novice programmers \cite{eckerdal}. 

Given Marton et al.'s rubric for effective patterns of variation, and the identification of ``critical features,'' one can discern between more or less theoretically effective examples of the object or concept given to a student to learn. On this basis, Luxton-Reilly et al. \cite{Luxton13} suggest that identifying distinct clusters of solutions can help instructors select appropriate examples of code for helping students learn. They also suggest that it is helpful for teachers' own understanding and quality of feedback and guidance. Facilitating the discovery or identification of critical features, which are possibly both teacher-specific and task-specific, is a major challenge I will address in this thesis. 

Other pedagogical strategies that involve comparing and contrasting examples have also been shown to have learning benefits. The pedagogical method of comparing and contrasting ways of approaching a solution has now been validated in the literature of mathematics education research \cite{Star07}, cognitive science \cite{loewenstein2003analogical,kurtz01learning,telling}, and computing education research \cite{Suhonen08, PatitsasICER13}. Peer reviews and assessments, surveyed in \cite{peerReview98}, are yet another opportunity for students to learn from compare and contrast.

\subsection{Synthesizing Knowledge Across Sources and Modalities}

The synthesis of understanding derived from multiple sources is critical to journalism and humanities scholarship and in technical fields, like mathematics.

\subsubsection{Humanities Scholarship and Journalistic Analysis}  
Wineburg \cite{wineburg} shows how students of history come to their understanding of complex events. One important behavior is the students' use of multiple simultaneous documents to understand context. Wineburg finds that ``\ldots context is everything \ldots who wrote something; what their political view is; what the situation in the world is at that moment \ldots you need to see the situation from many points-of-view \ldots''

Software has recently been built to help scholars and journalists analyze and synthesize knowledge across sources. For example, the AP's Overview Project is an example of software designed to help journalists analyze thousands of documents. Similarly, WordSeer \cite{wordseer} allows scholars in the humanities to make sense of a corpus of relevant texts by providing the ability to look at multiple sources and do textual analysis of the content. Crowdlines \cite{luther} employed crowd-sourcing to help people learn and synthesize information from diverse online sources. Since humans are skilled at evaluating high-level structure and making connections between sources, crowdworkers created outlines for important topics that included diverse perspectives from multiple document sources. 

Shahaf et al. \cite{shahaf} created algorithms for ``information cartography,'' algorithmically sub-sampling large collections of documents on a common topic and laying them out as a 2-D map of interrelated documents for users to explore and read. This technique has been applied to scholarly publications and news articles on complex current events, such as the European debt crisis or the Israeli-Palestinian conflict. These domains contain interrelated parallel story-lines that evolve and intersect over time, and are represented as such in the resulting ‘Metro Maps’ of information.

\subsubsection{Science, Technology, Engineering and Math (STEM)} 
The simple question, ``What machine learning books are accessible and appropriate for my high school-aged daughter?'' recently kicked off a very lively discussion on a corporate engineering mailing list. It is a question that humans with a model of the learner, e.g., high school student, and the subject matter, e.g., machine learning, can answer well. Jardine \cite{jardine} generated reading lists specifically for novices hoping to become experts in a particular area, using a personalized pagerank function and Latent Topic Models. 

Educational psychologists have found that multiple explanations within and across multiple modalities can help students learn mathematical problem solving. Both Tabachneck et al. \cite{Tabachneck} and Cox and Brna \cite{cox} found that students fared better at problem solving when using multiple strategies and/or representations, such as diagrams, written algebra, tables, and natural language. Ainsworth points out that giving students an opportunity to consider different representations may help them overcome the weaknesses of any particular representation.

This is also supported by authors of Metacademy.com, a popular online resource for teaching yourself machine learning: ``A good general piece of advice is to consult multiple resources. Different textbooks or courses will explain something from a different perspective \ldots [O]ften when reading one, you get an ‘aha!’ moment for something which didn't make sense in the other. Unfortunately, this option might not be practical unless you have access to a university library.'' \cite{metacademy}

\subsection{User Interface Design}
Tufte pioneered a layout technique called ``small multiples,'' designed to help viewers make rapid decisions about a wide array of items or variables: ``Small multiple designs \ldots answer directly by visually enforcing \ldots the differences among objects, \ldots the scope of alternatives.'' %We took inspiration from both of these layouts when designing the flowing grid layout for DocMatrix. 

%The common term sidebar was inspired by Hearst's faceted browsing \cite{facets}. But rather than display facets derived from metadata about each document, we extracted the common terms from a source closer to the content of the documents themselves: the tables of contents. Clicking on any of the terms in this list exposes the tables of contents, with relevant chapter titles highlighted; the actual terms contained in the tables of contents are the key to this ontological alignment.

Grokker is a document-clustering visualization system, with small popup windows to read texts in parallel \cite{slaney}. Unlike DocMatrix, Grokker's primary representation of a corpus of documents is as clusters of dots, but the study design and results are still relevant here. The task for Grokker readers was to quickly browse a large document collection, and then answer a set of questions to test their understanding. A key finding of this study was that small details of document viewability and the amount of time it took the participants to access content dramatically affected how much they understood about the domain.  In other words, small changes in the amount of time to switch between related documents was an important variable.  


\section{Methods for Analyzing Solutions}

While the methods described in this section all analyze solutions written in code, either as individuals or as a collection, some focus on supervised classification of solutions using pre-defined schemas and some focus on unsupervised clustering of code within collections; and some focus on explicitly identifying variation.

\subsection{Methods for Supervised Classification of Solutions}

Taherkhani and collaborators have developed several algorithm recognition methods. Two methods feature prominently in their work: (1) a method that creates a feature vector for the \emph{predefined target solution} based on roles of variables, beacons, and various other software metrics and (2) a method that scans solutions for \emph{predefined schemas} that are associated with algorithms of interest. In \cite{taherkhani13}, Taherkhani et al. have combined the two approaches into a more robust version which only compares software metrics and beacons on the code that have the same schema. While the software metrics used in these approaches are relevant to my thesis, the use of predefined schemas and target solutions is distinct from my approach of purely mining student solutions.

\subsection{Methods for Unsupervised Clustering of Code}

Unsupervised clustering encompasses everything from identifying plagiarism within a collection of solutions submitted by different people to loosely grouping solutions with similar approaches to solving a problem.

\subsubsection{Clone Detection}
There are multiple types of clones. The simplest clone is an exact copy. A parameter-substituted clone only differs from its copy by the values of identifiers and literals. A structure-substituted clone is a copy with something new swapped in for an entire subtree of the syntax tree. This is of particular interest to teachers looking for evidence of plagiarism within or across class offerings and for software engineers performing re-factoring who want to make their code more compliant with the DRY ("Do Not Repeat Yourself") principle of software development.

\citet{tiarks2011extended} does... \todo{expand on what I said earlier: The most powerful algorithm recognition methods can identify clones that are modified beyond just structural substitutions...}

A popular plaigerism detection algorithm, MOSS, uses \todo{winnowed?} document fingerprinting by \citet{schleimer2003winnowing}. A fingerprint is constructed by computing hashes for all $n$-grams in a document (for some chosen $n$). Similar code will contain similar fingerprint components. \todo{it also does parameter substitution, right? and what's this winnowing business? did document fingerprinting exist before this citation?}

%These latter two types of clones are difficult to identify even with current state-of-the-art techniques \cite{taherkhani12, taherkhani13}. 

\subsection{Methods for Describing Solution Variation}

\begin{comment}
Braun and Clarke \cite{thematic06} argue that its application to qualitative data outside psychological research is justified. It is in direct contrast to methods in which a hypothesis or theory is first declared, and then evidence for and against it is gathered from the data. 
\end{comment}

\citet{Luxton13} used thematic analysis to capture the variation between correct solutions in their dataset. Thematic analysis comes out of the field of psychology as a way to build theory from observing patterns in organic, free-form statements from subjects, or, in this case, submissions for coding assignments. They aimed to discover the kinds and degree of variation between student-generated solutions that fulfilled the specifications of short introductory Java programming exercises. By thematic analysis of student submissions, the authors generated a taxonomy that captured the variation between correct solutions in their dataset. They then created an Eclipse plug-in for classifying new code examples based on their taxonomy. 

\section{Methods for Analyzing Paths to Solutions}

Using automated classification methods, \citet{Piech} found distinct development paths students take to achieve working solutions that fulfilled the specifications of short introductory programming exercises in Java. Students' incremental paths were classified by pipeline that included milestone discovery, Hidden Markov Modeling of the students' process, and clustering of solution paths. These identified paths are visualized as finite state machine transition diagrams. The evaluation focused on predicting midterm exam grades and detecting milestone difficulty.

\citet{sudol12} also classify and map out distinct paths to solutions to introductory programming exercises. After using a Markov Model to generate a ``problem state graph,'' the authors applied their Probabilistic Distance to Solution (PDS) metric to the graph to estimate the number of states between an observed program model and the model of a correct solution.

Kiesmueller et al. \cite{Kiesmueller} attempted to recognize strategies at a very high level, which are not specific to the challenge at hand. Example high-level problem-independent strategies were a top-down or bottom-up programming style. Helminen et al. \cite{ICERHelminen} introduced novel interactive graphs for examining the problem solving process of students working on small programming-like problems. However, problems with multiple solutions were outside the scope of their investigation.

%\section{Introduction}
\section{Thesis Proposal}

%I partitioned the related work based on method of discovering the space of solutions and existing structures within learning environments that this knowledge could enhance. The specific domain to which these methods and interventions are applied is noted as each reference is discussed. 

%The work highlighted in the first section below addresses the feasibility and current state-of-the-art in classifying student solutions and solution paths, using machine learning algorithms and visualization methods. The second section highlights work from the Computer Science Education community on the relevance of solution space knowledge to various methods used within learning environments. %The final subsection surveys the existing domains in which these methods have been applied, specifically concerning the type and scale of programming challenges.

Since terminology across research domains can vary, I will define the terms in which I will describe previous research and my own: 
\begin{itemize}
\item A solution is code that a particular person wrote in response to a prompt or problem description.
\item Solution clusters represent different patterns of implementation. For example, there may be two distinct solution clusters, both achieving the same input-output behavior but by different means. 
\item A solution path is a series of code snapshots generated while a person is working toward meeting a particular input-output behavior specification. 
\item The ``space of student solutions'' refers to the aggregation of student-generated solutions and the solution clusters they form.
\end{itemize}

\subsection{Comparing and Contrasting Examples}


Marton et al.'s variation theory \cite{Marton13} holds that in order to learn something, one must see examples that vary along particular dimensions: ``contrast,'' as in pairing it with something it is not; ``generalization,'' as in presenting multiple instances of the object or concept to be learned, varying only that which is irrelevant; ``separation,'' as in presenting multiple instances of the object or concept, varying only that which can vary internally without changing the object or concept into something else; and ``fusion,'' as in seeing multiple examples in which previously analytically separated aspects must be processed together to recognize the object or concept. The aspects which are related to these dimensions of variation and therefore define the object or concept are called ``critical features.''

Peer reviews and assessments, surveyed in \cite{peerReview98}, are one of the existing pedagogies in which teachers ask students to compare and constrast examples. The pedagogical method of comparing and contrasting ways of approaching a solution has now been validated in the literature of mathematics education research \cite{Star07}, cognitive science \cite{loewenstein2003analogical,kurtz01learning,telling}, and computing education research \cite{Suhonen08, PatitsasICER13}.

Given Marton et al.'s rubric for effective patterns of variation, and the identification of ``critical features,'' one can discern between more or less theoretically effective examples of the object or concept given to a student to learn. On this basis, Luxton-Reilly et al. \cite{Luxton13} suggest that identifying distinct clusters of solutions can help instructors select appropriate examples of code for teaching purposes. 

While this research has focused on the effects of multiple, varying examples on student learning, it is also, as suggested by Luxton-Reilly et al. \cite{Luxton13}, helpful for teachers' own understanding and quality of feedback and guidance. Facilitating the discovery or identification of critical features, which are possibly both teacher-specific and task-specific, is a major challenge I will address in this thesis.

\subsection{Feature Engineering}

In order to discover or identify critical features, it is necessary to generate a set of candidate features. A variety of methods have been employed in the literature to select or engineer features, but those which I describe here are focused on features that humans can perceive by looking at the actual code submission.

\subsubsection{Abstract Syntax Trees, Dependency Graphs, and Control Flow Graphs}

Aggarwal et al. \cite{aggarwalprinciples} address the critical nature of feature engineering in the context of machine-learning-based automated grading. The grading rubric is based on the authors' understanding of how humans perceive and assess programs. They describe how humans first look for certain ``signature features,'' such as certain control structures, dependencies, and keywords. If the necessary features are in place, then more fine-grained assessment can be made. Are the correct structures used, and are statements ordered properly? This increasingly fine-grained assessment can continue on, to include terminating conditions and dependencies between data structures, until the human is satisfied. 

Aggarwal et al. suggest extracting these features from a code submission's Abstract Syntax Tree (AST), Control Flow Graph, Data Dependence Graph, and/or Program Dependence Graph. The authors assert that these graphs will be helpful even if the code represents only a partial solution. Note that these features are targeted at labeling each solution with a numerical score based on correctness, not on disguishing between equally correct solutions representing different approaches to a problem.

%, human considerations are weighed heavily in their feature design.

\subsubsection{Thematic Analysis}

Instead of grading submissions based on a rubric of human acceptability and correctness, Luxton-Reilly et al. \cite{Luxton13} used thematic analysis to capture the variation between correct solutions in their dataset. Thematic analysis comes out of the field of psychology as a way to build theory from observing patterns in organic, free-form statements from subjects, or, in this case, submissions for coding assignments. Braun and Clarke \cite{thematic06} argue that its application to qualitative data outside psychological research is justified. It is in direct contrast to methods in which a hypothesis or theory is first declared, and then evidence for and against it is gathered from the data. 

Luxton-Reilly et al. \cite{Luxton13} aim to discover the kinds and degree of variation between student-generated solutions that fulfilled the specifications of short introductory Java programming exercises. By thematic analysis of student submissions, the authors generated a taxonomy that captures the variation between correct solutions in their dataset. They then created an Eclipse plug-in for classifying new code examples based on their taxonomy. 

\subsubsection{Compiler Concepts}

Some of the Java exercises in the corpus studied by Luxton-Reilly et al. \cite{Luxton13} were as simple as writing a function which takes two integers and returns their sum. Within these simple tasks, variation was still found in the way students used parentheses, declared and initialized variables, and made assignments. In order to use established, non-ambiguous terms for their observations, Luxton-Reilly et al. reference compiler concepts, such as tokens, classes of tokens, and control flow graphs. 

They labeled types of variations as structural, syntactic, or relating to presentation. The control flow graphs represent structural variation. The nodes of control flow graphs are blocks of code that have a single entry point, single exit point, and no internal branching. The flow between blocks of code is represented by the edges connecting the control graph nodes. If the control graph (structure) of two solutions is the same, then the syntactic variation within those blocks of code are compared by looking at the sequence of token classes. Presentation-based variation, such as variable names and spacing, is only examined when two solutions are structurally and syntactically the same.

Luxton-Reilly et al.'s Eclipse plugin takes as input a collection of Java source files and creates a library of structurally unique Java code, which becomes the categories of files. On their corpus, they found a small number of highly populated categories, which did not always line up with the category of the instructor's implementation.

\subsubsection{Adding Input-Output Behavior}

Huang et al. \cite{MOOCshop} considered tens of thousands of solutions submitted to Stanford's Fall 2011 Machine Learning MOOC, and identified clusters of solutions based on measures of syntactic but also functional similarity. The syntactic similarity was the edit distance between solutions' ASTs, using the tree edit distance function described in Shasha et al. \cite{shasha1994exact}. Code submissions were also grouped by their success or failure on a battery of unit tests (input-output behavior). By pairing behavioral descriptions with structure-based distance measures between submissions, the authors got a fine-grained breakdown of submissions, across correctness and structure.

\subsubsection{Program Comprehension}

Yet another field is relevant when looking for the internal design variation of identically behaving solutions. Over the course of several papers, culminating in their most recent article \cite{taherkhani13}, Taherkhani et al. have drawn from the field of program comprehension to develop several algorithm recognition methods. Two methods feature prominently in their work: (1) a method that creates a feature vector for the \emph{predefined target solution} based on roles of variables, beacons, and various other software metrics and (2) a method that scans solutions for \emph{predefined schemas} that are associated with algorithms of interest. In \cite{taherkhani13}, Taherkhani et al. have combined the two approaches into a more robust version which only compares software metrics and beacons on the code that have the same schema. While the software metrics used in these approaches are relevant to my thesis, the use of predefined schemas and target solutions does not fit in with my approach of purely mining student solutions.

\subsubsection{Clone and Plaigiarism Detection}

Recognizing algorithms is similar to clone detection, which is also associated with plaigiarism detection. There are multiple types of clones. The simplest clone is an exact copy of another section of code. A parameter-substituted clone only differs from its copy by the values of identifiers and literals. A structure-substituted clone is a copy with something new swapped in for an entire subtree of the syntax tree. The most sweeping algorithm recognition methods can identify clones that are modified beyond just structural substitutions \cite{tiarks2011extended}. These latter two types of clones are difficult to identify with current state-of-the-art techniques \cite{taherkhani12, taherkhani13}. 

One strategy which has worked well for discovering similar code segments in larger pieces of source code is document fingerprinting \cite{schleimer2003winnowing}, where functions of small sections of code are considered ``fingerprints.'' A fingerprint is constructed by computing hashes for all $n$-grams in a document (for some chosen $n$). Similar code will contain similar fingerprint components. These fingerprints are potential features for classifying or clustering code.

%%In their algorithm for the MOSS plagiarism-detection system, Schleimer et al. \cite{schleimer2003winnowing} introduce winnowing, an efficient technique for generating fingerprints with a guarantee that identical code sections above a user-selected minimum size will always yield identical components in the fingerprint.


%We might instead put aside not only the idea of winnowing, but of selecting a compact fingerprint at all: while a fingerprint might normally be constructed by computing hashes for all $n$-grams in a document (for some chosen $n$) and selecting a subset of them, we can retain all the hashes. If we believe that dissimilar code will contain different $n$-grams in different amounts, we can use vectors in the space of $n$-grams to identify submissions. In our investigations with Java source code, however, even after removing irrelevant features such as variable names and considering token-level $n$-grams, the number of different $n$-grams was too large for unsupervised clustering to be effective.

%\textbf{Make plugin description concrete and accurate}

\subsection{Mathematical Modeling Applied to Solutions}

While critical features are chosen based on their ability to enhance teachers' understanding or guide the choice of example solutions, it is also necessary to select features that support classification or clustering. Specifically, it is necessary to select features that support classification or clustering that is understandable and acceptable to the human in the loop.

%\subsubsection{Supervised ML}
%
%Taherkhani et al. \cite{taherkhani12} demonstrated the practicality of identifying which sorting algorithm a student implemented, using supervised machine learning methods. \textbf{Details! Features? Methods?}
%
%\subsubsection{Unsupervised ML}
%
%%The most recent relevant work on finding clusters in solutions comes from Stanford. Huang et al. \cite{MOOCshop} consider tens of thousands of solutions submitted to Stanford's Fall 2011 Machine Learning MOOC, and identify clusters of solutions based on measures of syntactic and functional similarity. They make a case for mapping out the solution space using analysis beyond just input-output behavior. They claim that output-based feedback alone is insufficient, since the relationship between input-output pairs and bugs, both mental and programmatic, is not a one-to-one mapping. For students' approaches to implementing regularized logistic regression, similarly behaving programs were implemented in significantly different ways. 

\subsubsection{Active and Interactive Machine Learning}

On his interactive machine learning (IML) course webpage, Dr. Brad Knox describes IML as ``machine learning with a human in the learning loop, observing the result of learning and providing input meant to improve the learning outcome.'' Active learning is a subset of semi-supervised machine learning in which the algorithm can query the human in the loop. Active/interactive machine learning techniques, which can take advantage of human experts in the loop to resolve uncertainties, have been deployed for de-duplication in Stonebraker et al.'s Data Tamer \cite{stonebraker2013data} and in a cardiac ECG-based alarm system \cite{JWiensNIPS}. The work on applying interactive machine learning to the educational context is not as mature, but actively being pursued. For example, Basu et al. \cite{basupowergrading} have simulated an interactive machine learning framework for helping teachers grade large numbers of clustered free response textual answers.

%\textbf{Describe HCI aspects of IML in survey paper assigned by Knox?}

\subsubsection{Probabilistic Approaches}

Using automated classification methods, Piech et al. \cite{Piech} found distinct development paths students take to achieve working solutions that fulfilled the specifications of short introductory programming exercises in Java. Students' incremental paths were classified by pipeline that included milestone discovery, Hidden Markov Modeling of the students' process, and clustering of solution paths. These identified paths are visualized as finite state machine transition diagrams. The evaluation focused on predicting midterm exam grades and detecting milestone difficulty. 

Sudol et al.'s new metric, Probabilistic Distance to Solution, and its successful application to introductory programming exercises, is a second example of the feasibility of classifying and mapping out distinct paths to solutions \cite{sudol12}. After using a Markov Model to generate a ``problem state graph,'' the authors applied their Probabilistic Distance to Solution (PDS) metric to the graph to estimate the number of states between an observed program model and the model of a correct solution.

\subsubsection{Learning Students' Process and Behavior}

The following examples highlight research that is further from relevance to this thesis because the paths to a working solution are classified by behavior rather than the type of final solution found. Kiesmueller et al. \cite{Kiesmueller} attempted to recognize strategies at a very high level, which are not specific to the challenge at hand. Example high-level problem-independent strategies were a top-down or bottom-up programming style. Helminen et al. \cite{ICERHelminen} introduced novel interactive graphs for examining the problem solving process of students working on small programming-like problems. However, problems with multiple solutions were outside the scope of their investigation.



%
%\subsubsection{Active Learning of Solution Clusters}
%
%CITE DATA TAMER APPLICABILITY FOR ITS HUMAN ACTIVE LEARNING COMPONENT to partial solutions
%Community source the distance metrics/cluster boundaries: Ask Student: "Is this what you did?" Ask teacher: "Are these the same?"

%\subsection{Relevant Learning Environment Methods}




%More concretely, comparing and contrasting solution approaches Patitsas et al. \cite{PatitsasICER13} has 
%
% These methods have their theoretical basis in variation theory \cite{} and social cognitive theory \cite{}. 
%
%Comparing and contrasting dierent solution approaches is
%known in math education and cognitive science to increase
%student learning { what about CS? In this experiment, we
%replicated work from Rittle-Johnson and Star, using a pretest{
%intervention{posttest{follow-up design (n=241). Our intervention was an in-class workbook in CS2. A randomized half
%of students received questions in a compare-and-contrast
%style, seeing dierent code for dierent algorithms in parallel. The other half saw the same code questions sequentially,
%and evaluated them one at a time. Students in the former
%group performed better with regard to procedural knowledge (code reading & writing), and 
%exibility (generating,
%recognizing & evaluating multiple ways to solve a problem).
%The two groups performed equally on conceptual knowledge.
%Our results agree with those of Rittle-Johnson and Star, indicating that the existing work in this area generalizes to CS
%education.
%
%In light of these pedagogical frameworks, Luxton-Reilly et al. \cite{Luxton13} suggest that identifying distinct clusters of solutions can help instructors select appropriate examples of code for teaching purposes.


%\textbf{insert transition to next section}
\subsection{Feedback to Students}

Peer-pairing can stand in place of staff assistance, to both reduce the load on teaching staff and give students a chance to gain ownership of material through teaching it to someone else. Weld et al. speculate about peer-pairing in MOOCs based on student competency measures \cite{WeldHcomp12}, and Klemmer et al. demonstrate peer assessments' scalability to large online design-oriented classes \cite{Klemmer}.

Generating tailored feedback to students in large classes tackling problems even as short as introductory programming assignments requires many man-hours of repetitive work. Singh et al. \cite{rishabh} are pushing the state of the art of automated feedback for short introductory programming assignments. However, their software is currently only differentiating between solutions based on their input-output characteristics. For example, this system cannot currently differentiate between two different sorting algorithms. If there are common dead-ends that have been identified by looking at incorrect student solutions to a particular problem, by hand, this system can identify that a student is very close to a known dead-end approach, but it cannot identify {\em which} functionally equivalent variant of a correct solution a student is approaching. 

Singh et al.'s automated feedback represents one end of the spectrum for providing tailored feedback to students because hints are algorithmically generated. Luxton-Reilly et al. \cite{Luxton13}, Huang et al. \cite{MOOCshop}, and Basu et al. \cite{basupowergrading} represent the other end, by ``force multiplying'' human-generated feedback or ``powergrading.'' By clustering syntactically similar solutions which fail on the same input-output tests, Huang et al. aim to enable the sending of appropriate teacher-written feedback to entire clusters of solutions. Basu et al. \cite{basupowergrading} focus their work on grading short textual free-response questions, but the idea of reducing the number of actions necessary for the expert labeler is the same.

%\subsection{Domain of Application}
%
%LOOK AT COMPARCH COMMUNITY, describe scale of Singh solution, MOOCshop solution, types of programming (languages)
\section{Readable Code}

\cite{6005readings,artofreadablecode,styleguides}
\section{OverCode}

There is a growing body of work on both the frontend and backend required to manage and present the large volumes of solutions gathered from MOOCs, intelligent tutors, online learning platforms, and large residential classes. The backend necessary to analyze solutions expressed as code has followed from prior work in fields such as program analysis, compilers, and machine learning. A common goal of this prior work is to help teachers monitor the state of their class, or provide solution-specific feedback to many students. However, there has not been much work on developing interactive user interfaces that enable a teacher to navigate the large space of student solutions. 

We first present here a brief review of the state of the art in the backend, specifically about analyzing code generated by students who are independently attempting to implement the same function. This will place our own backend in context. We then review the information visualization principles and systems that inspired our frontend contributions.

\subsection{Related Work in Program Analysis}

\subsubsection{Canonicalization and Semantics-Preserving Transformations}

When two pieces of code have different syntax, and therefore different abstract syntax trees (ASTs), they may still be semantically equivalent. A teacher viewing the code may want to see those syntactic differences, or may want to ignore them in order to focus on semantic differences. Semantics-preserving transformations can reduce or eliminate the syntactic differences between code. Applying semantics-preserving transformations, sometimes referred to as canonicalization or standardization, has been used for a variety of applications, including detecting clones \cite{baxter} and automatic ``transform-based diagnosis’’ of bugs in students’ programs written in programming tutors \cite{xutransformation}. 

OverCode also canonicalizes solutions, using variable renaming. OverCode’s canonicalization is novel in that its design decisions were made to maximize {\it human readability} of the resulting code. As a side-effect, syntactic differences between answers are also reduced.

\subsubsection{Abstract Syntax Tree-based Approaches}

Huang et al. \citeyear{MOOCshop} worked with short Matlab/Octave functions submitted online by students enrolled in a machine learning MOOC. The authors generate an AST for each solution to a problem, and calculate the tree edit distance between all pairs of ASTs, using the dynamic programming edit distance algorithm presented by Shasha et al. \citeyear{shasha1994exact}. Based on these computed edit distances, clusters of syntactically similar solutions are formed. The algorithm is quadratic in both the number of solutions and the size of the ASTs. Using a computing cluster, the Shasha algorithm was applied to just over a million solutions. 

Calculating tree-edit distances between all pairs of ASTs allows Huang et al. to analyze differences within each line. It’s also computationally expensive, with quadratic complexity both in the number of solutions and the size of the ASTs~\cite{MOOCshop}. The OverCode analysis pipeline does not reason about differences any finer than a line of code, but it has linear complexity in the number of solutions and in the size of the ASTs.

Codewebs \cite{codewebs} created an index of ``code phrases'' for over a million submissions from the same MOOC and semi-automatically identified equivalence classes across these phrases, using a data-driven, probabilistic approach. The Codewebs search engine accepts queries in the form of subtrees, subforests, and contexts that are subgraphs of an AST. A teacher labels a set of AST subtrees considered semantically meaningful, and then queries the search engine to extract all equivalent subtrees from the dataset. OverCode does analyze the AST of student solutions but only in order to reformat code and rename variables that behave similarly on a test case. All further code comparison is done through string matching lines of code that have consistent formatting and variable names.

Both Codewebs \cite{codewebs} and Huang et al. \citeyear{MOOCshop} use unit test results and AST edit distance to identify clusters of submissions that could potentially receive the same feedback from a teacher. These are non-interactive systems that require hand-labeling in the case of Codewebs, or a computing cluster in the case of Huang et al. In contrast, OverCode’s pipeline does not require hand-labeling and runs in minutes on a laptop, then presents the results in an interactive user interface.

\subsubsection{Supervised Machine Learning and Hierarchical Pairwise Comparison}

Semantic equivalence is another way of saying that two solutions have the same schema. A {\em schema}, in the context of programming, is a high-level cognitive construct by which humans understand or generate code to solve problems \cite{Soloway1984}. For example, two programs that implement bubble sort have the same schema, bubble sort, even though they may have different low-level implementations. Taherkhani et al. \citeyear{taherkhani12,taherkhani13} used supervised machine learning methods to successfully identify which of several sorting algorithms a solution used. Each solution is represented by statistics about language constructs, measures of complexity, and detected roles of variables. Variable roles are determined based on variable behavior. OverCode identifies common variables based on variable behavior as well. Both methods consider the sequence of values that variables are assigned to, but OverCode does not attempt to categorize variable behavior as one of a set of predefined roles. Similarly, Taherkhani et al.’s method can identify sorting algorithms that have already been analyzed and included in its training dataset. OverCode, in contrast, handles problems for which the algorithmic schema is not already known. 

Luxton-Reilly et al. \citeyear{Luxton13} label types of variations as structural, syntactic, or presentation-related. The structural similarity between solutions in a dataset is captured by comparing their control flow graphs. If the control flow of two solutions is the same, then the syntactic variation within the blocks of code is compared by looking at the sequence of token classes. Presentation-based variation, such as variable names and spacing, is only examined when two solutions are structurally and syntactically the same. In contrast, our approach is not hierarchical, and uses dynamic information in addition to syntactic information.

\subsubsection{Program Synthesis}

There has also been work on analyzing each student solution individually to provide more precise feedback. Singh et al. \citeyear{rishabh} use a constraint-based synthesis algorithm to find the minimal changes needed to make an incorrect solution functionally equivalent to a reference implementation. The changes are specified in terms of a problem-specific error model that captures the common mistakes students make on a particular problem.

Rivers and Koedinger \citeyear{riversaied} propose a data-driven approach to create a solution space consisting of all possible paths from the problem statement to a correct solution. To project code onto this solution space, the authors apply a set of normalizing program transformations to simplify, anonymize, and order the program’s syntax. The solution space can then be used to locate the potential learning progression for a student submission and provide hints on how to correct their attempt. Unlike OverCode’s variable renaming method, which reflects the most common names chosen by students, Rivers and Koedinger replace student variable names with arbitrary symbols, i.e. \codevar{daysInMonth} might be mapped to \codevar{v0}. 

Singh et al. and Rivers and Koedinger focus on providing hints to students along their path to a correct solution. Instead of providing hints, the aim of our work is to help instructors navigate the space of \emph{correct} solutions and therefore techniques based on checking only the functional correctness are not helpful in computing similarities and differences between such solutions.

\subsubsection{Code Comparison Tools}
File comparison tools, such as Apple FileMerge, Microsoft WinDiff, and Unix diff, are a class of tools that analyze and present differences between files. Highlighting indicates inserted, deleted, and changed text. Unchanged text is collapsed. Some of these tools are customized for analyzing code, such as Code Compare. They are also integrated into existing integrated development environments (IDE), including IntelliJ IDEA and Eclipse. These code-specific comparison tools may match methods rather than just comparing lines. Three panes side-by-side are used to show code during three-way merges of file differences. There are tools, e.g. KDiff3, which will show the differences between four files when performing a distributed version control merge operation, but that appears to be an upper limit. These tools do not scale beyond comparing a handful of programs simultaneously. OverCode can show hundreds or thousands of solutions simultaneously, and its visualization technique dims the lines that are shared with the most common solution, rather than using colors to indicate inserted or deleted lines.

MOSS~\cite{schleimer2003winnowing} is a widely used system for finding similarities across student solutions for detecting plagiarism. MOSS uses a windowing technique to select fingerprints from hashes of $k$-grams from a solution. It first creates an index mapping fingerprints to corresponding locations for all solutions. It then fingerprints each solution again to compute the list of matching fingerprints for the solution. Finally, it rank-orders the fingerprint matches by their size for each pair of solution match. This algorithm enables MOSS to find partial matches between two solutions that are in different positions with good accuracy. OverCode, on the other hand, uses a simple linear algorithm to create stacks of solutions with the same canonical form. It uses an equivalence based on the set of statements in a solution to capture position-independent statement matches.

\subsection{Related Work in User Interfaces for Solution Visualization}

Several user interfaces have been designed for providing grades or feedback to students at scale, and for browsing large collections in general, not just student solutions. 

Basu et al. \citeyear{basupowergrading} provide a novel user interface for {\it powergrading} short-answer questions. Powergrading means assigning grades or writing feedback to many similar answers at once. The backend uses machine learning that is trained to cluster answers, and the frontend allows teachers to read, grade or provide feedback to those groups of similar answers simultaneously. Teachers can also discover common misunderstandings. The value of the interface was verified in a study of 25 teachers looking at their visual interface with clustered answers. When compared against a baseline interface, the teachers assigned grades to students substantially faster, gave more feedback to students, and developed a ``high-level view of students’ understanding and misconceptions’’ \cite{basuDivideAndConquer}.

%Beyond powergrading, there is a variety of related work in the field of information visualization, which is focused on the visual presentation of data to aid human cognition. Flamenco\cite{flamenco} was a faceted browsing interface for a large collection of art. made use of faceted metadata, and was designed for exploring a large collection, which was a difficult task using traditional query-based interfaces. The interface was tested by 32 art history students to browse 35,000 images of art. OverCode is also about making a large collection easily browsable are comparable to Flamenco, though our collection is code submitted by students, and our metadata is produced by our program analysis pipeline.

At the intersection of information visualization and program analysis is Cody\footnote{\url{mathworks.com/matlabcentral/cody}}, an informal learning environment for the Matlab programming language. Cody does not have a teaching staff but does have a {\em solution map} visualization to help students discover alternative ways to solve a problem. A solution map plots each solution as a point against two axes: time of submission on the horizontal axis, and code size on the vertical axis, where \textit{code size} is the number of nodes in the parse tree of the solution. Despite the simplicity of this metric, solution maps can provide quick and valuable insight when assessing large numbers of solutions~\cite{ICERGlassman}.

% Originally launched in January of 2012, there are over 1500 problems posted by and for users, and users have submitted over 281,000 solutions. 
%  (rcm: took this out because it’s not obvious that this has anything to do with the solution map
% Though the information visualizations continue to be updated on the Cody website, screenshots from the site during the Summer of 2013 have been published \ref{ICERGlassman}.
%  (rcm: took this out because it’s not relevant -- people can go look at the Cody website, right?)

OverCode has also been inspired by information visualization projects like WordSeer \cite{wordseerlitcomp13,wordseercikm13} and CrowdScape \cite{crowdscape}. WordSeer helps literary analysts navigate and explore texts, using query words and phrases \cite{wordseerhcir11}. CrowdScape gives users an overview of crowd-workers’ performance on tasks. An overview of crowd-workers each performing on a task, and an overview of submitted code, each executing a test case, are not so different, from an information presentation point of view.
\section{Foobaz}
Foobaz builds upon past systems for enabling grading at scale, particularly in the context of teaching students how to program well. We also provide background on the principles of good variable naming.

\todo{add "What's in a name?" and binkley2011improving papers and The Java Programmer’s Phrase Book and Debugging Method Names; When a function is uncommented, a human reader’s program comprehension may depend almost entirely on variable names (Lawrie et al., ‘06)--talk about its experiment conclusions.}

\subsection{User Interfaces for Grading at Scale}
The powergrading paradigm \cite{basupowergrading} enables teachers to assign grades or write feedback to many similar answers at once. Their interface focused on powergrading for short-answer questions from the U.S. Citizenship exam. After machine learning clustered answers, the frontend allowed teachers to read, grade, or provide feedback on similar answers simultaneously. When compared against a baseline interface, the teachers assigned grades to students substantially faster, gave more feedback to students, and developed a ``high-level view of students' understanding and misconceptions'' \cite{basuDivideAndConquer}.

OverCode \cite{overcode} took steps toward enabling powergrading in the domain of programming education. The system enabled teachers to visualize and explore the thousands of student submissions to simple exercises in an introductory programming MOOC. OverCode used static and dynamic analysis to cluster similar solutions on the basis of variable behavior, and then presented these ``stacks'' to the teacher. It was found that the system enabled teachers to more quickly understand the different strategies and misconceptions used by students. Foobaz builds upon the OverCode pipeline, using the stacks and common variables it produces as the basis for delivering feedback on variable names at scale.

Foobaz presents a significant departure from OverCode. The Foobaz system uses the OverCode program analysis backend to bring to the fore what OverCode intentionally hid: variable names. In order to create the new user interface, we developed a technique for visualizing the variation of names within clusters. The feedback mechanism is also distinct. OverCode helped teachers write general feedback for the entire class, while Foobaz creates personalized feedback quizzes for each student.

Two more recently published systems help teachers give programming students subjective feedback on coding style: AutoStyle \cite{autostyle} and ACES \cite{ACES}. AutoStyle is designed for automatically composing code style feedback to programming students at scale. Style, in this system, refers to the effective use of programming idioms; it does not allow for feedback on variable names, indentation, or punctuation. ACES relies on static analysis, Abstract Syntax Trees, and unsupervised learning to streamline the process of grading on style. The analysis backend recommends feedback for each new submission based on past solutions and teacher annotations. However, in its user interface, the teacher still reviews submissions one at a time, ultimately limiting its ability to scale.

% Taherkhani - not sure where to fit in
% Taherkhani et al. [2012, 2013] \cite{} categorize variables based on variable behavior. Both methods consider the sequence of values to which variables are assigned. Unlike Taherkhani et al., OverCode does not attempt to categorize variable behavior as one of a set of predefined roles.


\subsection{Variable Name Design}
Designing names for variables is an art more than a science. Donald Knuth compares a good programmer to an essayist who, ``with thesaurus in hand, chooses the names of variables carefully and explains what each variable means'' \cite{literateprogramming}. Without modifying execution, names can express to the human reader the type and purpose of an object, as well as suggest the kinds of operators used to manipulate it \cite{operands}.

The freedom that programmers have when naming classes, functions, and variables allows them to name variables poorly. At best, bad variable names are the subject of humor, i.e., ``26 Variable Names for Busy Developers: a, b, c, d, e...'' \cite{hackeronion}. Various naming conventions, like Hungarian notation, have evolved to help developers use their freedom wisely. The Google C++ Style Guide authors assert that their most important consistency rules govern naming, which are arbitrary but consistent in order to increase human readability \cite{GoogleCStyleGuide}. 

One of MIT's largest core software engineering courses (6.005) \cite{UseGoodNames6.005CodeReviewReading} specifically recommends students use verb phrases for method names and noun phrases for variable and class names. The 6.005 staff also ask that student balance the need for descriptive and meaningful names with conciseness. Finally, abbreviations are considered bad form; they can be difficult to unpack for both native and non-native English speakers.

Programmers can develop their own heuristics for good variable names through the experiential learning process of building, debugging, and sharing increasingly large programs with others and their future selves. During interviews, one professor explained an elaborate set of guidelines that she personally developed and teaches to her students that are specific to the domain in which she works, i.e., machine learning [Finale Doshi, personal communication].

\subsection{Variable Names in Classrooms}
Bad variable names can throw roadblocks into the paths of already struggling beginners. Introductory programming students will, for example, iterate over the elements of an array but name the iterator as if it is an index, and vice versa [Guttag, personal communication]. It could be an innocent mistake that lengthens debugging time or indicative of a flawed mental model. Rapid feedback on variable names may remind students why naming matters, correct their flawed mental models, and expose them to examples of teacher-endorsed naming conventions and styles \cite{ieeeRapidFeedback}.




\section{Learnersourcing Personalized Hints}

Our work builds on prior research on delivering personalized support to students. It is also informed by existing research on the pedagogical benefits of reflection and explanation.

\subsection{Personalized Support}

Several types of solutions have been deployed to help students get the personalized attention they need. These solutions span the spectrum from recruiting more teaching assistants from the ranks of previous students \cite{communityTAs} to automating hints using intelligent tutoring systems. 

Intelligent tutoring systems can provide personalized hints and other assistance to each student based on a pre-programmed student model. For example, previous systems sought to provide support through the use of adaptive scripts \cite{kumar2007tutorial}, or cues from the student’s problem-solving actions \cite{diziol}. Despite the advantage of automated support, intelligent tutoring systems often require domain experts to design and build them, making them expensive to develop. \todo{Include ``current ITSs require an exact formalization of the underlying domain knowledge
which is usually a substantial amount of work: researchers have reported 100-1000 hours
of authoring time needed for one hour of instruction [MBA03] from `Feedback Provision Strategies in Intelligent Tutoring
Systems Based on Clustered Solution Spaces'''} Furthermore, domain experts who generate these hints may also suffer from the ``curse of knowledge’’: the difficulty experts have when trying to see something from a novice’s point of view \cite{curse}. 

Unlike intelligent tutoring systems, the HelpMeOut system \cite{helpmeout} does not require a pre-programmed student model. It assists programmers during their debugging by suggesting code modifications mined from debugging performed by previous programmers. However, the suggestions lack explanations in plain language unless they are added by experts (teachers), so the limits imposed by the time, expense, and curse of knowledge of experts still apply.

Discussion forums derive their value from the content produced by the teachers and students who use them. These systems can harness the benefits of peer learning, where students can benefit from generating and receiving help from each other. However, as the system has no student model, the information is available to all students whether or not it is ultimately relevant. Students can receive personalized attention only if they post a question and receive a response. 

\subsection{Reflection and Explanation}
In this work, we aim to design opportunities for students to help others while simultaneously reflecting on their own solutions. Existing theories indicate that reflection is a critical method for triggering the transformation from conflict and doubt into clarity and coherence \cite{dewey1933}. Turning that reflection into a self-explanation further improves understanding \cite{selfexplanation}. According to Turns et al. \cite{asee}, the absence of reflection in traditional engineering education is a significant shortcoming. 

Novices may become confused if asked to reflect on their solution or compare it to a fellow student’s solution; this is not necessarily bad for learning outcomes. Piaget theorized that cognitive disequilibrium, experienced as confusion, could trigger learning due to the creation or restructuring of knowledge schema \cite{disequilibrium}. D’Mello et al. maintain that confusion can be productive, as long as it is both appropriately injected and resolved \cite{productiveconfusion}. 

Similarly, reflecting on a peer’s conceptual development or alternative solution may bring about cognitive conflict that prompts reevaluation of the student’s own beliefs and understanding \cite{kavanagh}. As such, peer instruction \cite{mazur} and peer assessment \cite{peerassessment} have not only been integrated into many classroom activities, but have also formed the basis of several online systems for peer-learning. For example, Talkabout organizes students into discussion groups based on characteristics such as gender or geographic balance \cite{talkabout}.

Recent work on learnersourcing proposes that learners can collectively generate educational content for future learners while engaging in a meaningful learning experience themselves \cite{kim2013learnersourcing,weir2015,mitros2015}. For example, Crowdy enables people to annotate how-to videos while simultaneously learning from the video \cite{weir2015}. Beyond existing work, we investigate alternatives for what support students should be prompted to provide, based on their own work as well as the needs of their peers. We also explore several ways to trigger productive reflection as a byproduct of hint creation, by prompting students to either self-reflect or compare their own solutions to those produced by peers. 



Also consider related work folder(s) on machine, Zotero




