\section{Learnersourcing Personalized Hints}

Our work builds on prior research on delivering personalized support to students. It is also informed by existing research on the pedagogical benefits of reflection and explanation.

\subsection{Personalized Support}

Several types of solutions have been deployed to help students get the personalized attention they need. These solutions span the spectrum from recruiting more teaching assistants from the ranks of previous students \cite{communityTAs} to automating hints using intelligent tutoring systems. 

Intelligent tutoring systems can provide personalized hints and other assistance to each student based on a pre-programmed student model. For example, previous systems sought to provide support through the use of adaptive scripts \cite{kumar2007tutorial}, or cues from the student’s problem-solving actions \cite{diziol}. Despite the advantage of automated support, intelligent tutoring systems often require domain experts to design and build them, making them expensive to develop. \todo{Include ``current ITSs require an exact formalization of the underlying domain knowledge
which is usually a substantial amount of work: researchers have reported 100-1000 hours
of authoring time needed for one hour of instruction [MBA03] from `Feedback Provision Strategies in Intelligent Tutoring
Systems Based on Clustered Solution Spaces'''} Furthermore, domain experts who generate these hints may also suffer from the ``curse of knowledge’’: the difficulty experts have when trying to see something from a novice’s point of view \cite{curse}. 

Unlike intelligent tutoring systems, the HelpMeOut system \cite{helpmeout} does not require a pre-programmed student model. It assists programmers during their debugging by suggesting code modifications mined from debugging performed by previous programmers. However, the suggestions lack explanations in plain language unless they are added by experts (teachers), so the limits imposed by the time, expense, and curse of knowledge of experts still apply.

Discussion forums derive their value from the content produced by the teachers and students who use them. These systems can harness the benefits of peer learning, where students can benefit from generating and receiving help from each other. However, as the system has no student model, the information is available to all students whether or not it is ultimately relevant. Students can receive personalized attention only if they post a question and receive a response. 

\subsection{Reflection and Explanation}
In this work, we aim to design opportunities for students to help others while simultaneously reflecting on their own solutions. Existing theories indicate that reflection is a critical method for triggering the transformation from conflict and doubt into clarity and coherence \cite{dewey1933}. Turning that reflection into a self-explanation further improves understanding \cite{selfexplanation}. According to Turns et al. \cite{asee}, the absence of reflection in traditional engineering education is a significant shortcoming. 

Novices may become confused if asked to reflect on their solution or compare it to a fellow student’s solution; this is not necessarily bad for learning outcomes. Piaget theorized that cognitive disequilibrium, experienced as confusion, could trigger learning due to the creation or restructuring of knowledge schema \cite{disequilibrium}. D’Mello et al. maintain that confusion can be productive, as long as it is both appropriately injected and resolved \cite{productiveconfusion}. 

Similarly, reflecting on a peer’s conceptual development or alternative solution may bring about cognitive conflict that prompts reevaluation of the student’s own beliefs and understanding \cite{kavanagh}. As such, peer instruction \cite{mazur} and peer assessment \cite{peerassessment} have not only been integrated into many classroom activities, but have also formed the basis of several online systems for peer-learning. For example, Talkabout organizes students into discussion groups based on characteristics such as gender or geographic balance \cite{talkabout}.

Recent work on learnersourcing proposes that learners can collectively generate educational content for future learners while engaging in a meaningful learning experience themselves \cite{kim2013learnersourcing,weir2015,mitros2015}. For example, Crowdy enables people to annotate how-to videos while simultaneously learning from the video \cite{weir2015}. Beyond existing work, we investigate alternatives for what support students should be prompted to provide, based on their own work as well as the needs of their peers. We also explore several ways to trigger productive reflection as a byproduct of hint creation, by prompting students to either self-reflect or compare their own solutions to those produced by peers. 
