When I was in elementary school, I wanted a pet. Specifically, I wanted a pet robot that could learn from human interaction. Nothing fancy, just some associative learning so that I could pat it on the head while saying its name and it's software would eventually figure out that that particular string of syllables referred to itself.

My dad told me that one way to create this hypothetical learning pet robot was to start with its brain. To do that, I'd need to learn how to program. I still remember the thrill of accomplishment after sitting down together with a Student Version of Matlab for three whole hours on a Saturday afternoon, stepping through for loops, printing variable values and iterating through arrays, figuring out how to harness the the computer one programming construct at a time. If the program was wrong, I could change the file of instructions and run it again--over and over again if necessary, until it worked the way I wanted it to.

By ninth grade, I'd finally finished programming my hypothetical pet robot's ear. By tenth grade, I'd made a lot of progress toward programming its mouth. However, by then, I'd also outgrown the original idea of a learning pet robot.

I haven't outgrown the idea that programming is deeply creative and incredibly powerful. Someone who programs can autonomously control objects in the physical world and manipulate abstract ideas, like the likelihood of observations given some assumptions. Programming is the biggest force multiplier of human effort humanity has ever developed.

At MIT and Stanford, so many students take a programming course that it is practically a general institute requirement. Outside traditional schools, students can learn programming online from organizations like edX and Kahn Academy or from developer bootcamps that promise a good paying job after less than six months of intense training. New York City and the US Federal Government have both recently announced initiatives designed to expose every student to programming, or at least computational thinking, before they graduate from high school.

I was lucky; I got one-on-one personal tutoring, which is considered the gold-standard in terms of educational outcomes \cite{bloom}. The glut of students trying to learn programming is inflating class sizes into the hundreds or thousands at schools like MIT, Stanford and Berkeley. The rapid, personalized feedback and attention that is possible within a one-on-one tutoring relationship becomes prohibitively painful or expensive. To me, the obvious answer to handling the growing demand for programming education is, of course, also programming. The only remaining question is ``how."