% -*-latex-*-
% $Log: cover.tex,v $
% Revision 1.7  2001/02/08 18:53:16  boojum
% changed some \newpages to \cleardoublepages
%
% Revision 1.6  1999/10/21 14:49:31  boojum
% changed comment referring to documentstyle
%
% Revision 1.5  1999/10/21 14:39:04  boojum
% *** empty log message ***
%
% Revision 1.4  1997/04/18  17:54:10  othomas
% added page numbers on abstract and cover, and made 1 abstract
% page the default rather than 2.  (anne hunter tells me this
% is the new institute standard.)
%
% Revision 1.4  1997/04/18  17:54:10  othomas
% added page numbers on abstract and cover, and made 1 abstract
% page the default rather than 2.  (anne hunter tells me this
% is the new institute standard.)
%
% Revision 1.3  93/05/17  17:06:29  starflt
% Added acknowledgements section (suggested by tompalka)
% 
% Revision 1.2  92/04/22  13:13:13  epeisach
% Fixes for 1991 course 6 requirements
% Phrase "and to grant others the right to do so" has been added to 
% permission clause
% Second copy of abstract is not counted as separate pages so numbering works
% out
% 
% Revision 1.1  92/04/22  13:08:20  epeisach
\addcontentsline{toc}{chapter}{Cover page}
\title{Clustering and Visualizing Solution Variation In Massive Programming Classes}
\todo{Massive removed based on rob's feedback on NEML poster}
\author{Elena L. Glassman}
\department{Department of Electrical Engineering \\ and Computer Science}
% If the thesis is for two degrees simultaneously, list them both
% separated by \and like this:
 \degree{Doctor of Philosophy}
%\degree{Bachelor of Science in Computer Science and Engineering}
\degreemonth{August}
\degreeyear{2016}
\thesisdate{August 7, 2016}

%% By default, the thesis will be copyrighted to MIT.  If you need to copyright
%% the thesis to yourself, just specify the `vi' documentclass option.  If for
%% some reason you want to exactly specify the copyright notice text, you can
%% use the \copyrightnoticetext command.  
\copyrightnoticetext{\copyright ~Elena L. Glassman, 2016.}

\begin{comment}
\copyrightnoticetext{\copyright ~Elena L. Glassman, 2016. \\
{\footnotesize
This work is licensed under the Creative
Commons Attribution-NonCommercial 2.5
License. To view a copy of this license, visit
\url{http://creativecommons.org/licenses/by-nc/2.5/} or send a
letter to
Creative Commons
543 Howard Street, 5th Floor,
San Francisco, California, 94105, USA.
}}
\end{comment}

% If there is more than one supervisor, use the \supervisor command
% once for each.
\supervisor{Robert C. Miller}{Professor of Electrical Engineering \\ and Computer Science}

% This is the department committee chairman, not the thesis committee
% chairman.  You should replace this with your Department's Committee
% Chairman.
%\chairman{Daniel Blankschtein}{Chairman, Department Committee on Graduate Students}
\chairman{------------------}{Chairman, Department Committee on Graduate Students}

% Make the titlepage based on the above information.  If you need
% something special and can't use the standard form, you can specify
% the exact text of the titlepage yourself.  Put it in a titlepage
% environment and leave blank lines where you want vertical space.
% The spaces will be adjusted to fill the entire page.  The dotted
% lines for the signatures are made with the \signature command.
\maketitle

% The abstractpage environment sets up everything on the page except
% the text itself.  The title and other header material are put at the
% top of the page, and the supervisors are listed at the bottom.  A
% new page is begun both before and after.  Of course, an abstract may
% be more than one page itself.  If you need more control over the
% format of the page, you can use the abstract environment, which puts
% the word "Abstract" at the beginning and single spaces its text.

%% You can either \input (*not* \include) your abstract file, or you can put
%% the text of the abstract directly between the \begin{abstractpage} and
%% \end{abstractpage} commands.

% First copy: start a new page, and save the page number.
\cleardoublepage
% Uncomment the next line if you do NOT want a page number on your
% abstract and acknowledgments pages.
\pagestyle{empty}
\setcounter{savepage}{\thepage}
\begin{abstractpage}\addcontentsline{toc}{chapter}{Abstract}
% $Log: abstract.tex,v $
% Revision 1.1  93/05/14  14:56:25  starflt
% Initial revision
%
% Revision 1.1  90/05/04  10:41:01  lwvanels
% Initial revision
%
%
%% The text of your abstract and nothing else (other than comments) goes here.
%% It will be single-spaced and the rest of the text that is supposed to go on
%% the abstract page will be generated by the abstractpage environment.  This
%% file should be \input (not \include 'd) from cover.tex.

In massive programming-based engineering courses, a single exercise may yield thousands of student solutions. Some solutions are superficially different, while others differ in a fundamental way. Understanding large-scale variation in solutions is a hard but important problem. For teachers, this variation can be a source of pedagogically valuable examples and expose corner cases not yet covered by autograding. For students, the variation in a large class means that other students may have struggled along a similar solution path, hit the same bugs, and can offer hints based on that earned expertise.

This thesis describes three systems that explore the value of solution variation in large-scale programming and simulated digital circuit classes. All three systems have been evaluated using data or live deployments in on-campus or edX courses with thousands of students. (1) OverCode visualizes thousands of programming solutions using static and dynamic analysis to cluster similar solutions. It lets teachers quickly develop a high-level view of student understanding and misconceptions and provide feedback that is relevant to many student solutions. (2) Foobaz clusters variables in student programs by their names and behavior so that teachers can give feedback on variable naming. Rather than requiring the teacher to comment on thousands of students individually, Foobaz generates personalized quizzes that help students evaluate their own names by comparing them with good and bad names from other students. (3) ClassOverflow collects and organizes solution hints indexed by the autograder test that failed or a performance characteristic like size or speed. It helps students reflect on their debugging or optimization process, generates hints that can help other students with the same problem, and could potentially bootstrap an intelligent tutor tailored to the problem.

These systems demonstrate how clustering and visualizing student solutions helps teachers and students provide types of one-on-one design feedback at scale that was previously only possible through one-on-one tutoring or in a small classroom. They also demonstrate ways that teachers can systematically curate good and bad design choices from trends and outliers generated by an entire population of hundreds or thousands of students. The feedback generated by both teachers and students can be re-used by future students who attempt the same problem.

\begin{comment}

helps teachers respond to class-population-level trends and outliers in student designs, and curate pedagogically examples that can then These systems demonstrate how, 

Foobaz and ClassOverflow demonstrate how practices that previously could only occur in one-on-one interactions or wtihin small classrooms can be scaled up to serve thousands of current and future students. User testing of OverCode and its extensions demonstrate how visualizing and clustering large collections of code can help teachers gain insight into and take advantage of the space of student-generated solutions in ways that were not possible in smaller classrooms.

Massive programming courses produce a massive collection of student solutions for each programming exercise. The solutions to any particular problem vary along many dimensions, including bugs, naming, syntax, and semantics. The distribution of solutions along these dimensions reflect students' prior knowledge, the teacher's course curriculum and explanations so far, and misconceptions common to all programming students.

Personal tutors can respond to an individual tutee's solution, and only draw on their personal recollection of previously observed solution variation. Teachers teaching larger groups of students can directly observe that a significant fraction of students are struggling with a particular concept or implementation, and respond appropriately with rapid contextual feedback. They can also pick out particular student solutions as examples to illustrate different concepts or ways of solving a problem, rather than solely relying on their own creativity to composing these examples from scratch. 

When scaling up to massive programming courses, it becomes painful or prohibitively exhausting to engage with student solutions this way. It is also an opportunity: teachers have access to a comparatively dense sampling of the distribution over student solution bugs, naming, syntax, and semantics. 

The systems in this thesis help teachers take advantage of the massive collections of solutions, enabling either (1) the same teaching practices that were previously only tractable in smaller courses or (2) new practices that are only possible when a massive collection of student solutions are available. The common empowering mechanism is clustering and visualizing solution variation for human understanding. Using these systems, teachers can gain insights into student design choices, detect autograder failures, process solutions that deserve partial credit, use targeted learnersourcing to collect hints for other students, and give personalized style feedback at scale.
\end{comment}

\end{abstractpage}

% Additional copy: start a new page, and reset the page number.  This way,
% the second copy of the abstract is not counted as separate pages.
% Uncomment the next 6 lines if you need two copies of the abstract
% page.
% \setcounter{page}{\thesavepage}
% \begin{abstractpage}
% % $Log: abstract.tex,v $
% Revision 1.1  93/05/14  14:56:25  starflt
% Initial revision
%
% Revision 1.1  90/05/04  10:41:01  lwvanels
% Initial revision
%
%
%% The text of your abstract and nothing else (other than comments) goes here.
%% It will be single-spaced and the rest of the text that is supposed to go on
%% the abstract page will be generated by the abstractpage environment.  This
%% file should be \input (not \include 'd) from cover.tex.

In massive programming-based engineering courses, a single exercise may yield thousands of student solutions. Some solutions are superficially different, while others differ in a fundamental way. Understanding large-scale variation in solutions is a hard but important problem. For teachers, this variation can be a source of pedagogically valuable examples and expose corner cases not yet covered by autograding. For students, the variation in a large class means that other students may have struggled along a similar solution path, hit the same bugs, and can offer hints based on that earned expertise.

This thesis describes three systems that explore the value of solution variation in large-scale programming and simulated digital circuit classes. All three systems have been evaluated using data or live deployments in on-campus or edX courses with thousands of students. (1) OverCode visualizes thousands of programming solutions using static and dynamic analysis to cluster similar solutions. It lets teachers quickly develop a high-level view of student understanding and misconceptions and provide feedback that is relevant to many student solutions. (2) Foobaz clusters variables in student programs by their names and behavior so that teachers can give feedback on variable naming. Rather than requiring the teacher to comment on thousands of students individually, Foobaz generates personalized quizzes that help students evaluate their own names by comparing them with good and bad names from other students. (3) ClassOverflow collects and organizes solution hints indexed by the autograder test that failed or a performance characteristic like size or speed. It helps students reflect on their debugging or optimization process, generates hints that can help other students with the same problem, and could potentially bootstrap an intelligent tutor tailored to the problem.

These systems demonstrate how clustering and visualizing student solutions helps teachers and students provide types of one-on-one design feedback at scale that was previously only possible through one-on-one tutoring or in a small classroom. They also demonstrate ways that teachers can systematically curate good and bad design choices from trends and outliers generated by an entire population of hundreds or thousands of students. The feedback generated by both teachers and students can be re-used by future students who attempt the same problem.

\begin{comment}

helps teachers respond to class-population-level trends and outliers in student designs, and curate pedagogically examples that can then These systems demonstrate how, 

Foobaz and ClassOverflow demonstrate how practices that previously could only occur in one-on-one interactions or wtihin small classrooms can be scaled up to serve thousands of current and future students. User testing of OverCode and its extensions demonstrate how visualizing and clustering large collections of code can help teachers gain insight into and take advantage of the space of student-generated solutions in ways that were not possible in smaller classrooms.

Massive programming courses produce a massive collection of student solutions for each programming exercise. The solutions to any particular problem vary along many dimensions, including bugs, naming, syntax, and semantics. The distribution of solutions along these dimensions reflect students' prior knowledge, the teacher's course curriculum and explanations so far, and misconceptions common to all programming students.

Personal tutors can respond to an individual tutee's solution, and only draw on their personal recollection of previously observed solution variation. Teachers teaching larger groups of students can directly observe that a significant fraction of students are struggling with a particular concept or implementation, and respond appropriately with rapid contextual feedback. They can also pick out particular student solutions as examples to illustrate different concepts or ways of solving a problem, rather than solely relying on their own creativity to composing these examples from scratch. 

When scaling up to massive programming courses, it becomes painful or prohibitively exhausting to engage with student solutions this way. It is also an opportunity: teachers have access to a comparatively dense sampling of the distribution over student solution bugs, naming, syntax, and semantics. 

The systems in this thesis help teachers take advantage of the massive collections of solutions, enabling either (1) the same teaching practices that were previously only tractable in smaller courses or (2) new practices that are only possible when a massive collection of student solutions are available. The common empowering mechanism is clustering and visualizing solution variation for human understanding. Using these systems, teachers can gain insights into student design choices, detect autograder failures, process solutions that deserve partial credit, use targeted learnersourcing to collect hints for other students, and give personalized style feedback at scale.
\end{comment}

% \end{abstractpage}

\cleardoublepage

\section*{Preface}\addcontentsline{toc}{chapter}{Preface}
I am not a professional software developer, but learning how to write programs in middle school was one of the most empowering skills my father could ever have taught me. I did not need access to chemicals or heavy machinery. I only needed a computer and occasionally an internet connection. I acquired datasets and wrote programs to extract interesting patterns in them. It was and still is a creative outlet.

More recently, the value of learning how to code, or at least how to think computationally, has gained national attention. Last September, New York City Mayor Bill de Blasio announced that all public schools in NYC will be required to offer computer science to all students by 2025. In January of this year, the White House released its Computer Science for All initiative, "offering every student the hands--on computer science and math classes that make them job--ready on day one" (President Obama, 2016 State of the Union Address).

The economic value and employment prospects associated with knowing how to program, the subsiding stigma of being a computer "geek" or "nerd," and the production and popularity of movies about programmers may be responsible for driving up enrollment in computer science classes to unprecedented levels. Hundreds or thousands of students enroll in programming courses at schools like MIT, Stanford, Berkeley and the University of Washington.

One-on-one tutoring is considered a gold standard in education, and programming education is likely no exception. However, most students are not going to receive that kind of personalized instruction. We, as a computer science community, may not be able to offer one-on-one tutoring at a massive scale, but can we create systems that enhance the teacher and student experiences in massive classrooms in ways that would never have been possible in one-on-one tutoring? This thesis is one particular approach to answering that question. %are not even possible in just a one-on-one environment?% How can we best teach the next generation of students how to code when there are so many of them?

\begin{comment}
When I was in elementary school, I wanted a pet. Specifically, I wanted a pet robot that could learn from human interaction. Nothing fancy, just some associative learning so that I could pat it on the head while saying its name and its software would eventually figure out that that particular sequence of syllables referred to itself.

My dad told me that one way to create this hypothetical learning pet robot was to start with its brain. To do that, I'd need to learn how to program. I still remember the thrill of accomplishment after sitting down together with a Student Version of Matlab for three whole hours on a Saturday afternoon, stepping through for loops, printing variable values and iterating through arrays, figuring out how to harness the the computer one programming construct at a time. If the program was wrong, I could change the file of instructions and run it again--over and over again if necessary, until it worked the way I wanted it to.

By ninth grade, I'd finally finished programming my hypothetical pet robot's ear. By tenth grade, I'd made a lot of progress toward programming its mouth. However, by then, I'd also outgrown the original idea of a learning pet robot.

I haven't outgrown the idea that programming is deeply creative and incredibly powerful. Someone who programs can autonomously control objects in the physical world and manipulate abstract ideas, like the likelihood of observations given some assumptions. Programming is the biggest force multiplier of human effort humanity has ever developed.

At MIT and Stanford, so many students take a programming course that it is practically a general institute requirement. Outside traditional schools, students can learn programming online from organizations like edX and Kahn Academy or from developer bootcamps that promise a good paying job after less than six months of intense training. New York City and the US Federal Government have both recently announced initiatives designed to expose every student to programming, or at least computational thinking, before they graduate from high school.

I was lucky; I got one-on-one personal tutoring, which is considered a gold standard in terms of educational outcomes \cite{bloom}. The glut of students trying to learn programming is inflating class sizes into the hundreds or thousands at schools like MIT, Stanford and Berkeley. The rapid, personalized feedback and attention that is possible within a one-on-one tutoring relationship becomes prohibitively painful or expensive. To me, the obvious answer to handling the growing demand for programming education is, of course, also programming. The only remaining question is ``how."
\end{comment}

\cleardoublepage

\section*{Acknowledgments}\addcontentsline{toc}{chapter}{Acknowledgments}

Thank you, thank you, thank you to my advisor, Rob Miller, and my other co-authors on this thesis work, Rishabh Singh, Jeremy Scott, Philip Guo, Lyla Fischer, Aaron Lin, and Carrie Cai. The past and present members of our User Interface Design group were instrumental in helping me feel at home in a new area and get up to speed while having fun. I’m also grateful to my past internship mentors at Google and MSR, specifically Dan Russell, Merrie Ringel Morris, and Andrés Monroy-Hernández, who gave me the chance to try out my chops in new environments. I also want to thank my friends outside MIT, especially the greater Boston community of wrestling coaches, who helped me learn important lessons outside the classroom. And, last but not least, my partner Victor, my parents, and my brother, who have each supported me in their own way, from hugs, to proof-reading papers, to being technical sounding-boards, and finally, to demonstrating, as my brother has, how to switch fields, pick up a whole new set of skills, and look good doing it.

%I thank my labmates, Carrie, Juho, Tom, Katrina, ... \todo{finish}


\begin{comment}
There are countless individuals who have helped me in ways, large and small, complete this particular marathon. 

I thank my father for inviting me to think with him about interesting problems and principles in electrical engineering and computer science (even while we were in the middle of a Saturday morning jog through town when I was growing up), for teaching me how to program, for helping me learn new concepts by explaining them to him, and for providing emotional support throughout my time becoming an engineer. 

I thank my mother for helping me put together science fair boards at the last minute, copy-editing my paper drafts, and listening to my many updates about life and drama at the Institute.

I thank my brother for demonstrating how to completely switch fields of expertise, become a computer scientist and software developer, and look good doing it.

I thank my partner, Victor, for making my final year of graduate school so much fun, giving me encouraging pep-talks when I was felt down, and reminding me to "squeeze limes" and do what needs to be done, even when I did not want to.
\end{comment}

\begin{comment}


First, I would like to thank my parents for fostering my love of scientific inquiry (literally, a love of asking good questions and problems worth working on), for encouraging me to sit by the fire in a comfy chair and ask `what if?' and for proofreading all my camera-ready submissions. 

I would also like to thank my brother, for being a model of courage and steady progress on a PhD in a new field. We both switched fields to get to where we are, and making progress side by side (virtually) has been a resevoir of quiet support.

I am grateful for my thesis advisor Rob Miller's guidance, support, and insight. Bumping into him in the Gates tower stairway was one of the best things that happened to me in graduate school. 

My collaborators...

My lab neighbors...

My friends outsid the lab...



I am indebted to many people who both directly and indirectly
contributed to this thesis.  First, I would like to thank those
collaborators who directly contributed.  Most of all, I'm grateful
for the help and friendship of Mark Styczynski. Mark was my
collaborator on all matters computational for the past four years
--- his influence is evident throughout this document.  I'm also
grateful for my collaboration with Christopher Loose, who performed
many of the experiments on antimicrobial peptides described in
Chapter~\ref{chapter:amps}.  Finally, I would like to thank Isidore Rigoutsos, who
straddled the line between collaborator and advisor.  Isidore taught
me an attention to detail and a penchant for the UNIX command line
and vi editor.

Most importantly, I am indebted to Greg Stephanopoulos, my advisor,
whose guidance and support was unwavering these past six years. Greg
is the perpetual optimist --- always positive in the face of my many
failures along the way.  He also gave me the freedom to pursue
projects of my own choosing, which contributed greatly to my
academic independence, if not the selection of wise projects.

I am much obliged to my thesis committee members: my advisor Greg,
Isidore, Bill Green, and Bob Berwick.  My committee was always
flexible in scheduling and judicious in their application of both
carrots and sticks.

There are numerous people who contributed indirectly to this thesis.
First among these is my intelligent, lovely, and vivacious wife
Kathryn Miller--Jensen.  Next, my parents Carl and Julie, my sister
Heather, and my in-laws, Ron, Joyce, Suzanne, Jeff, and Mindi.
Finally, there are innumerable friends who contributed and to whom I
am greatly appreciative including Michael Raab, Joel Moxley, Bill
Schmitt, Vipin Guptda, and Jatin Misra.
\end{comment}

%%%%%%%%%%%%%%%%%%%%%%%%%%%%%%%%%%%%%%%%%%%%%%%%%%%%%%%%%%%%%%%%%%%%%%
% -*-latex-*-
