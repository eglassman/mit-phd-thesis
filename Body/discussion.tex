\chapter{Discussion}\label{chapter:discussion}

The systems in this thesis give teachers more awareness about the content generated by students in large programming classes and enable styles of teaching that are usually only feasible in smaller classes, such as discussions of variation and style that are directly driven by what the students have already written. These systems also to scale up automated compare-contrast, self-explanation, and formative assessment-style exercises whose content is generated by students and curated by teachers. \todo{add citations for value of formative assessment to related work} The current state of the art in theories of how humans learn predict that these supported interactions between teachers and students will enhance learning.


\section{Design Decisions and Choices}

%As the complexity of code increases, students face more design decisions.

This thesis work began with the vision that a teacher would someday be able to look at a display summarizing hundreds or thousands of student solutions to the same problem and immediately see---and comment on---good and bad design decisions that students made. The number of possible distinct student solutions grows rapidly with the number of design decisions and design choices students can make. 

The solution space could be imagined as a large n-dimensional space where each solution has a single coordinate. Each design decision, e.g., whether to solve a problem iteratively or recursively, would be a dimension in the solution space. The choices students make could be thought of as discrete points along that dimension in solution space. While the number of distinct combinations of design choices students choose can be large, the number of dimensions in this space, i.e., the underlying design decisions, grows much more slowly. 

However, the structure of this hypothetical solution space ignores how design choices affect each other. Some design decisions are mutually exclusive, e.g., looping over a particular array with a for or a while, some decisions are correlated with one another, and some decisions are completely independent. A tree-like description of the solution space can capture some of these relationships between decisions. If a node represents a design choice, e.g., to loop over an array, there may be two or more choices, e.g., a \texttt{for} or a \texttt{while} loop, that can be represented as child nodes. The choice to use a \texttt{for} loop poses an additional design design, e.g., whether to use \texttt{range} or \texttt{xrange}: \texttt{for i in [x]range(input)}. The average depth of the tree would correspond to the average number of design decisions the students faced and the average branching factor would correspond to the average number of distinct choices students in the corpus make at each decision point.

The curse of dimensionality predicts that, as we add more and more dimensions to the solution space, the density of solutions will decrease and the liklihood that any two solutions occupy the same location in that space will go down. The regularity of code discussed in the chapter on related work should help ameliorate the curse of dimensionality, but only partly. In other domains, it is often necessary to collect more data as the dimensionality of the space increases. However, given that the solutions are generated by students, the number of correct solutions are more likely to go down rather than up as the complexity of solutions, and the associated dimensionality of the solution space, increases.

This difficulties posed by high-dimensional spaces have been addressed, in each problem tackled, by choosing what information to ignore and, in some cases, what information to index by. For example, OverCode ignores white space, comments, variable names, and statement orders. It indexes by normalized lines. In order to assign a variable name quiz, Foobaz looks at the behavior of the variables in the student solution, ignoring everything else about it. Dear Beta and GroverCode forget what values cause a solution to fail a test case, preserving only that the test case has failed. Dear Gamma ignores circuit topology and indexes by the number of transistors.

It is important to note that this work is not intended to capture an enumeration of all possible design choices and resulting solutions. It is intended to capture the design choices students are actually making. The relative popularity of these choices is discussed in the section that follows.

\section{Capturing the Underlying Distribution}

There will be some design decisions within each solution that are rare and some that are common, some that exemplify good programming practices and others that do not. These decisions might create inefficient solutions, reveal a student's fundamental misunderstanding, or use a feature of the language in a creative way. The distribution of solutions along these dimensions of variation may reflect student prior knowledge, teacher explanations, and common misconceptions. 

During Hannah Wallach's invited talk at the interpretable machine learning workshop at ICML 2016, she made the following observation: computer scientists are often looking for better ways to find needles in haystacks and computationally-minded social scientists are trying to characterize the haystack. One could think of work like Codex~\cite{codex} and Webzeitgeist~\cite{webzeitgeist} as mechanisms for finding needles in haystacks through creative indexing of Ruby code and webpages, respectively. OverCode, Foobaz, Dear Beta, Dear Gamma, and GroverCode are trying to faithfully represent the haystack itself, while also supporting needle-finding. 

\section{Writing Solutions Well}

The focus on introductory Python programming courses is often just correctness. The MIT EECS introductory Python programming course whose staff used GroverCode makes it a policy not to penalize students on how their solution is written. This means that they sometimes have to hand out full marks to a solution that makes them groan.

This policy may exist for several reasons. The first is the clarity of the policy: if it is correct, it gets full credit. Second is the clarity of the evaluation: if it passes the suite of test cases, it is correct. Third is the apparent appropriateness of the objective for novices: it may be too hard for novices to write a solution well in addition to achieving correctness. % worry about writing well in addition to writing correctly.% bear both the cognitive load of writing a solution well in addition to correctly.

However, it can also be very hard to achieve a correct solution if it is not written well. Something as simple as poor variable names, such as giving an array index a name that suggests it holds the value of the array at that index, can cause students to produce incorrect code.

Just as there is no silver bullet for writing prose well, there is no silver bullet for writing solutions well. Solutions, prose, products, buildings etc., are all designed, and each community has its own ways for how to help students make good design decisions. For example, Steven Pinker's book ``The Sense of Style'', he suggests and then demonstrates how to pick apart examples of good writing to understand what makes them good~\cite{pinkersense}. Students of a particular design form may attend design studios, where they discuss each of their work in turn and give and receive constructive criticism from peers. In both these examples of prose and design, the designed objects are examined individually and also as a group, emulating the conditions for learning from variation espoused by the theories of learning reviewed in the related work. 

Perhaps because software can be marvelously complex, there is less of a design studio culture for software. Many software companies compose and maintain prescriptive style guides against which new code is carefully compared. These guides are not necessarily built on data about how software engineers actually design their code. 

While peer review is not a design studio, they are closer to that mode of education. Peer review practices are now being used in large online courses in order to make up for small teacher-to-student ratios. Some residential software engineering courses, e.g., MIT's 6.005, also set up infrastructure for peer review, even when the number of staff members is sufficient. This forces students to engage with some (usually random) sampling of how other students solved a problem. 

However, the work in this thesis takes this idea farther. Rather than hope that the randomly assigned peer review experiences provides students with a sufficient variety of examples, the work in this thesis attempts to pull out the design dimensions as well as concrete examples along those dimensions and, when possible, ask students to engage with them in a targeted, personalized way. 

This may be helpful even at the level of introductory programming. For example, according to variation theory, a student will understand the concept of a loop in a more generalized, robust way if they have seen all the different ways in which their peers have written that loop. The teacher can quickly and easily provide an expert perspective by commenting on the popular and rare choices. Students can be overwhelmed by choices. For example, they might ask themselves, {\it Should I use \texttt{range} or \texttt{xrange}? Does it matter?} With concrete examples, teachers can help students identify what matters and what does not. 

Many students now taking introductory programming courses will take these skills with them to other majors. While computer science majors can acquire the skills of writing code well in more advanced software engineering classes, the lessons in introductory courses on writing code well may be the only ones non-majors ever get.


\section{Clustering and Visualizing Solution Variation}

OverCode is a form of unsupervised clustering. Clustering is a function of similiarity measures and mechanisms for grouping or splitting clusters of data points. There is no true correct answer, but there are distinct failure modes when using it in the context of teachers reviewing solutions. Two are most relevant in this work. First, the representation of the solution and/or the measure of similiarity between solutions can ignore, hide, or otherwise fail to capture what the teacher cares about. Second, when (1) the teacher cannot easily decide what a cluster means to them based on its members and (2) the clustering algorithm cannot explicitly communicate why the cluster exists, the teacher may not trust the clustering and may not feel comfortable using it for propagating feedback and grades back to students. This is exacerbated when the teacher discovers a member of a cluster that seems not to belong.

The OverCode clustering pipeline attempts to escape the first clustering failure mode: erasing distinctions that the teacher may care about. For reasons discussed in detail in the OverCode chapter, OverCode is designed to reveal to the teacher what their students' solutions actually look like, modulo white space and comments. These solutions are rendered for the teacher using the most common variable names and statement orders.\todo{check the statement order} To stay true to what students actually wrote, this process preserves syntactic differences. For example, when iteratively exponentiating \texttt{base} to an exponent, there are multiple ways to multiply an accummulating variable \texttt{result} by \texttt{base} and save the product back into \texttt{result}, such as \texttt{result *= base} and \texttt{result = result * base}. If the teacher just gave a lecture on common forms of syntactic sugar, OverCode will be sensitive to whether or not students use it. OverCode would allow teachers to observe whether students absorbed the lesson on syntactic sugar based on the way they write their subsequent solutions. The fact that even small differences in syntax creates separate clusters within the OverCode pipeline nearly ensures that any syntactic choices a teacher is interested in has been preserved and can be filtered for in the interface.

 %ignored during the OverCode pipeline %decide for themselves what they want to ignore and 

%\section{Visualization}

The second clustering failure mode--producing clusters that the teacher does not trust--is avoided in several complementary ways. First, there is a clear interpretation of what an OverCode cluster can and cannot include, based on its platonic solution. Specifically, all solutions in the cluster have the same set of lines as the platonic solution after normalizing variable names, standardizing formatting, and removing comments. Second, the differences between clusters is made clear by highlighting which lines make the non-reference clusters different from the reference cluster. Third, the OverCode filtering features and rewrite rules help teachers change their view of solutions into one that preserves the differences they are explicitly interested in and ignores those they are not interested in. Note that filtering by semantic choices may only be possible when the work on Bayesian modeling of these solutions becomes more mature. 

\todo{add "Users do not notice renamed variables unless the names are inappropriate or do not look like they are human generated. Variable names as corpus-wide unique identifiers of a particular variable behavior is externally inconsistent and has low learnability."}

In general, the interfaces in this thesis cluster complex objects and visualize those clusters so that there is little guesswork about what is included and excluded in a group and what the boundary between groups is. There are no outliers that are grouped with something else without explanation. Rather than losing faith in the clustering process, outliers can be used as the teacher sees fit to spur improvements either to the software infrastructure of the course, e.g., the input-output testing harness, or the examples students are asked to engage with. 

\section{Language Agnosticism}

It is not surprising that the evolving values of variables would carry significant semantic meaning in code written by students at the introductory level in languages like Python and Matlab, especially if the style of programming is procedural. This thesis confirms that within the context of introductory procedural Python programs. According to Taherkhani et al.~\cite{taherkhani2010recognizing}, variables carry useful semantic meaning in object-oriented and functional programming styles as well. Taherkhani et al.~\cite{}, Gulwani et al.~\cite{gulwani_fse14}, and ~\cite{sajaniemi2002empirical} have all authored semantic analyses of Java, C++, and Pascal programming languages, respectively, with a focus on variable behavior. One could think of it as semantic variable duck typing based on variable behavior. 

OverCode and Foobaz could also be applied to other more programming languages. For a language to be displayable in OverCode, one would need (1) a logger of variable values inside tested functions, (2) a variable renamer and (3) a formatting standardization script. For non-variable-centric languages like Haskell, other dynamic characteristics of execution would likely need to be tracked.

\section{Limitations}

The thesis presents a series of case studies about how to present the variety of programming solutions in a human-interpretable way and make use of it in pedagogically valuable, scalable ways. The methods described only work in a particular domain of solutions: those that are executable and solve the same programming problem. This excludes natural language, for example. These case studies embody the design principles espoused, but there is no validated unifying recipe by which a corpus of student solutions can be processed and used. Each corpus and set of teacher values were considered together in order to engineer a representation of solutions--both in the pipeline and in the user interface--that would empower teachers and benefit students. There are many specific technical limitations of the approaches described in the previous chapters. In the next chapter, the section on future work describes these limitations and suggests next steps.

I am not aware of any complex domain where this kind of feature engineering and design has been automated. General guidelines and trial-and-error are accepted practice. Before the resurgence of deep neural nets training on massive data sets, one could reasonable argue that carefully human-designed features are were responsible for a great deal of the performance of machine learning systems. Otherwise, systems fall into the failure mode of garbage in, garbage out.

\section{Design Recommendations}

While this thesis does not offer a unifying recipe, these are some design recommendations based on experience accumulated since the start of this thesis:
\begin{enumerate}
\item When possible to do with high confidence, propagate human-assigned labels to similar data points.% automatically by the interface, such as grading a single solution and finding out that now an additional 39 solutions will appropriately receive the same grade and feedback.
\item Avoid outliers within clusters at all costs, because they cause doubt and confusion. 
\end{enumerate}

