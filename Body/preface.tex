I am not a professional software developer, but learning how to write programs in middle school was one of the most empowering skills my father could ever have taught me. I did not need access to chemicals or heavy machinery. I only needed a computer and occasionally an internet connection. I acquired datasets and wrote programs to extract interesting patterns in them. It was and still is a creative outlet.

More recently, the value of learning how to code, or at least how to think computationally, has gained national attention. Last September, New York City Mayor Bill de Blasio announced that all public schools in NYC will be required to offer computer science to all students by 2025. In January of this year, the White House released its Computer Science for All initiative, "offering every student the hands--on computer science and math classes that make them job--ready on day one" (President Obama, 2016 State of the Union Address).

The economic value and employment prospects associated with knowing how to program, the subsiding stigma of being a computer "geek" or "nerd," and the production and popularity of movies about programmers may be responsible for driving up enrollment in computer science classes to unprecedented levels. Hundreds or thousands of students enroll in programming courses at schools like MIT, Stanford, Berkeley and the University of Washington.

One-on-one tutoring is considered a gold standard in education, and programming education is likely no exception. However, most students are not going to receive that kind of personalized instruction. We, as a computer science community, may not be able to offer one-on-one tutoring at a massive scale, but can we create systems that enhance the teacher and student experiences in massive classrooms in ways that would never have been possible in one-on-one tutoring? This thesis is one particular approach to answering that question. %are not even possible in just a one-on-one environment?% How can we best teach the next generation of students how to code when there are so many of them?

\begin{comment}
When I was in elementary school, I wanted a pet. Specifically, I wanted a pet robot that could learn from human interaction. Nothing fancy, just some associative learning so that I could pat it on the head while saying its name and its software would eventually figure out that that particular sequence of syllables referred to itself.

My dad told me that one way to create this hypothetical learning pet robot was to start with its brain. To do that, I'd need to learn how to program. I still remember the thrill of accomplishment after sitting down together with a Student Version of Matlab for three whole hours on a Saturday afternoon, stepping through for loops, printing variable values and iterating through arrays, figuring out how to harness the the computer one programming construct at a time. If the program was wrong, I could change the file of instructions and run it again--over and over again if necessary, until it worked the way I wanted it to.

By ninth grade, I'd finally finished programming my hypothetical pet robot's ear. By tenth grade, I'd made a lot of progress toward programming its mouth. However, by then, I'd also outgrown the original idea of a learning pet robot.

I haven't outgrown the idea that programming is deeply creative and incredibly powerful. Someone who programs can autonomously control objects in the physical world and manipulate abstract ideas, like the likelihood of observations given some assumptions. Programming is the biggest force multiplier of human effort humanity has ever developed.

At MIT and Stanford, so many students take a programming course that it is practically a general institute requirement. Outside traditional schools, students can learn programming online from organizations like edX and Kahn Academy or from developer bootcamps that promise a good paying job after less than six months of intense training. New York City and the US Federal Government have both recently announced initiatives designed to expose every student to programming, or at least computational thinking, before they graduate from high school.

I was lucky. I got one-on-one personal tutoring, which is considered a gold standard in terms of educational outcomes \cite{bloom}. The glut of students trying to learn programming is inflating class sizes into the hundreds or thousands at schools like MIT, Stanford and Berkeley. The rapid, personalized feedback and attention that is possible within a one-on-one tutoring relationship becomes prohibitively painful or expensive. To me, the obvious answer to handling the growing demand for programming education is, of course, also programming. The only remaining question is ``how."
\end{comment}