%% This is an example first chapter.  You should put chapter/appendix that you
%% write into a separate file, and add a line \include{yourfilename} to
%% main.tex, where `yourfilename.tex' is the name of the chapter/appendix file.
%% You can process specific files by typing their names in at the 
%% \files=
%% prompt when you run the file main.tex through LaTeX.
\chapter{Related Work}\label{chapter:relatedwork}


%\section{Introduction}

\section{OverCode}

There is a growing body of work on both the frontend and backend required to manage and present the large volumes of solutions gathered from MOOCs, intelligent tutors, online learning platforms, and large residential classes. The backend necessary to analyze solutions expressed as code has followed from prior work in fields such as program analysis, compilers, and machine learning. A common goal of this prior work is to help teachers monitor the state of their class, or provide solution-specific feedback to many students. However, there has not been much work on developing interactive user interfaces that enable a teacher to navigate the large space of student solutions. 

We first present here a brief review of the state of the art in the backend, specifically about analyzing code generated by students who are independently attempting to implement the same function. This will place our own backend in context. We then review the information visualization principles and systems that inspired our frontend contributions.

\subsection{Related Work in Program Analysis}

\subsubsection{Canonicalization and Semantics-Preserving Transformations}

When two pieces of code have different syntax, and therefore different abstract syntax trees (ASTs), they may still be semantically equivalent. A teacher viewing the code may want to see those syntactic differences, or may want to ignore them in order to focus on semantic differences. Semantics-preserving transformations can reduce or eliminate the syntactic differences between code. Applying semantics-preserving transformations, sometimes referred to as canonicalization or standardization, has been used for a variety of applications, including detecting clones \cite{baxter} and automatic ``transform-based diagnosis’’ of bugs in students’ programs written in programming tutors \cite{xutransformation}. 

OverCode also canonicalizes solutions, using variable renaming. OverCode’s canonicalization is novel in that its design decisions were made to maximize {\it human readability} of the resulting code. As a side-effect, syntactic differences between answers are also reduced.

\subsubsection{Abstract Syntax Tree-based Approaches}

Huang et al. \citeyear{MOOCshop} worked with short Matlab/Octave functions submitted online by students enrolled in a machine learning MOOC. The authors generate an AST for each solution to a problem, and calculate the tree edit distance between all pairs of ASTs, using the dynamic programming edit distance algorithm presented by Shasha et al. \citeyear{shasha1994exact}. Based on these computed edit distances, clusters of syntactically similar solutions are formed. The algorithm is quadratic in both the number of solutions and the size of the ASTs. Using a computing cluster, the Shasha algorithm was applied to just over a million solutions. 

Calculating tree-edit distances between all pairs of ASTs allows Huang et al. to analyze differences within each line. It’s also computationally expensive, with quadratic complexity both in the number of solutions and the size of the ASTs~\cite{MOOCshop}. The OverCode analysis pipeline does not reason about differences any finer than a line of code, but it has linear complexity in the number of solutions and in the size of the ASTs.

Codewebs \cite{codewebs} created an index of ``code phrases'' for over a million submissions from the same MOOC and semi-automatically identified equivalence classes across these phrases, using a data-driven, probabilistic approach. The Codewebs search engine accepts queries in the form of subtrees, subforests, and contexts that are subgraphs of an AST. A teacher labels a set of AST subtrees considered semantically meaningful, and then queries the search engine to extract all equivalent subtrees from the dataset. OverCode does analyze the AST of student solutions but only in order to reformat code and rename variables that behave similarly on a test case. All further code comparison is done through string matching lines of code that have consistent formatting and variable names.

Both Codewebs \cite{codewebs} and Huang et al. \citeyear{MOOCshop} use unit test results and AST edit distance to identify clusters of submissions that could potentially receive the same feedback from a teacher. These are non-interactive systems that require hand-labeling in the case of Codewebs, or a computing cluster in the case of Huang et al. In contrast, OverCode’s pipeline does not require hand-labeling and runs in minutes on a laptop, then presents the results in an interactive user interface.

\subsubsection{Supervised Machine Learning and Hierarchical Pairwise Comparison}

Semantic equivalence is another way of saying that two solutions have the same schema. A {\em schema}, in the context of programming, is a high-level cognitive construct by which humans understand or generate code to solve problems \cite{Soloway1984}. For example, two programs that implement bubble sort have the same schema, bubble sort, even though they may have different low-level implementations. Taherkhani et al. \citeyear{taherkhani12,taherkhani13} used supervised machine learning methods to successfully identify which of several sorting algorithms a solution used. Each solution is represented by statistics about language constructs, measures of complexity, and detected roles of variables. Variable roles are determined based on variable behavior. OverCode identifies common variables based on variable behavior as well. Both methods consider the sequence of values that variables are assigned to, but OverCode does not attempt to categorize variable behavior as one of a set of predefined roles. Similarly, Taherkhani et al.’s method can identify sorting algorithms that have already been analyzed and included in its training dataset. OverCode, in contrast, handles problems for which the algorithmic schema is not already known. 

Luxton-Reilly et al. \citeyear{Luxton13} label types of variations as structural, syntactic, or presentation-related. The structural similarity between solutions in a dataset is captured by comparing their control flow graphs. If the control flow of two solutions is the same, then the syntactic variation within the blocks of code is compared by looking at the sequence of token classes. Presentation-based variation, such as variable names and spacing, is only examined when two solutions are structurally and syntactically the same. In contrast, our approach is not hierarchical, and uses dynamic information in addition to syntactic information.

\subsubsection{Program Synthesis}

There has also been work on analyzing each student solution individually to provide more precise feedback. Singh et al. \citeyear{rishabh} use a constraint-based synthesis algorithm to find the minimal changes needed to make an incorrect solution functionally equivalent to a reference implementation. The changes are specified in terms of a problem-specific error model that captures the common mistakes students make on a particular problem.

Rivers and Koedinger \citeyear{riversaied} propose a data-driven approach to create a solution space consisting of all possible paths from the problem statement to a correct solution. To project code onto this solution space, the authors apply a set of normalizing program transformations to simplify, anonymize, and order the program’s syntax. The solution space can then be used to locate the potential learning progression for a student submission and provide hints on how to correct their attempt. Unlike OverCode’s variable renaming method, which reflects the most common names chosen by students, Rivers and Koedinger replace student variable names with arbitrary symbols, i.e. \codevar{daysInMonth} might be mapped to \codevar{v0}. 

Singh et al. and Rivers and Koedinger focus on providing hints to students along their path to a correct solution. Instead of providing hints, the aim of our work is to help instructors navigate the space of \emph{correct} solutions and therefore techniques based on checking only the functional correctness are not helpful in computing similarities and differences between such solutions.

\subsubsection{Code Comparison Tools}
File comparison tools, such as Apple FileMerge, Microsoft WinDiff, and Unix diff, are a class of tools that analyze and present differences between files. Highlighting indicates inserted, deleted, and changed text. Unchanged text is collapsed. Some of these tools are customized for analyzing code, such as Code Compare. They are also integrated into existing integrated development environments (IDE), including IntelliJ IDEA and Eclipse. These code-specific comparison tools may match methods rather than just comparing lines. Three panes side-by-side are used to show code during three-way merges of file differences. There are tools, e.g. KDiff3, which will show the differences between four files when performing a distributed version control merge operation, but that appears to be an upper limit. These tools do not scale beyond comparing a handful of programs simultaneously. OverCode can show hundreds or thousands of solutions simultaneously, and its visualization technique dims the lines that are shared with the most common solution, rather than using colors to indicate inserted or deleted lines.

MOSS~\cite{schleimer2003winnowing} is a widely used system for finding similarities across student solutions for detecting plagiarism. MOSS uses a windowing technique to select fingerprints from hashes of $k$-grams from a solution. It first creates an index mapping fingerprints to corresponding locations for all solutions. It then fingerprints each solution again to compute the list of matching fingerprints for the solution. Finally, it rank-orders the fingerprint matches by their size for each pair of solution match. This algorithm enables MOSS to find partial matches between two solutions that are in different positions with good accuracy. OverCode, on the other hand, uses a simple linear algorithm to create stacks of solutions with the same canonical form. It uses an equivalence based on the set of statements in a solution to capture position-independent statement matches.

\subsection{Related Work in User Interfaces for Solution Visualization}

Several user interfaces have been designed for providing grades or feedback to students at scale, and for browsing large collections in general, not just student solutions. 

Basu et al. \citeyear{basupowergrading} provide a novel user interface for {\it powergrading} short-answer questions. Powergrading means assigning grades or writing feedback to many similar answers at once. The backend uses machine learning that is trained to cluster answers, and the frontend allows teachers to read, grade or provide feedback to those groups of similar answers simultaneously. Teachers can also discover common misunderstandings. The value of the interface was verified in a study of 25 teachers looking at their visual interface with clustered answers. When compared against a baseline interface, the teachers assigned grades to students substantially faster, gave more feedback to students, and developed a ``high-level view of students’ understanding and misconceptions’’ \cite{basuDivideAndConquer}.

%Beyond powergrading, there is a variety of related work in the field of information visualization, which is focused on the visual presentation of data to aid human cognition. Flamenco\cite{flamenco} was a faceted browsing interface for a large collection of art. made use of faceted metadata, and was designed for exploring a large collection, which was a difficult task using traditional query-based interfaces. The interface was tested by 32 art history students to browse 35,000 images of art. OverCode is also about making a large collection easily browsable are comparable to Flamenco, though our collection is code submitted by students, and our metadata is produced by our program analysis pipeline.

At the intersection of information visualization and program analysis is Cody\footnote{\url{mathworks.com/matlabcentral/cody}}, an informal learning environment for the Matlab programming language. Cody does not have a teaching staff but does have a {\em solution map} visualization to help students discover alternative ways to solve a problem. A solution map plots each solution as a point against two axes: time of submission on the horizontal axis, and code size on the vertical axis, where \textit{code size} is the number of nodes in the parse tree of the solution. Despite the simplicity of this metric, solution maps can provide quick and valuable insight when assessing large numbers of solutions~\cite{ICERGlassman}.

% Originally launched in January of 2012, there are over 1500 problems posted by and for users, and users have submitted over 281,000 solutions. 
%  (rcm: took this out because it’s not obvious that this has anything to do with the solution map
% Though the information visualizations continue to be updated on the Cody website, screenshots from the site during the Summer of 2013 have been published \ref{ICERGlassman}.
%  (rcm: took this out because it’s not relevant -- people can go look at the Cody website, right?)

OverCode has also been inspired by information visualization projects like WordSeer \cite{wordseerlitcomp13,wordseercikm13} and CrowdScape \cite{crowdscape}. WordSeer helps literary analysts navigate and explore texts, using query words and phrases \cite{wordseerhcir11}. CrowdScape gives users an overview of crowd-workers’ performance on tasks. An overview of crowd-workers each performing on a task, and an overview of submitted code, each executing a test case, are not so different, from an information presentation point of view.
\section{Foobaz}
Foobaz builds upon past systems for enabling grading at scale, particularly in the context of teaching students how to program well. We also provide background on the principles of good variable naming.

\todo{add "What's in a name?" and binkley2011improving papers and The Java Programmer’s Phrase Book and Debugging Method Names; When a function is uncommented, a human reader’s program comprehension may depend almost entirely on variable names (Lawrie et al., ‘06)--talk about its experiment conclusions.}

\subsection{User Interfaces for Grading at Scale}
The powergrading paradigm \cite{basupowergrading} enables teachers to assign grades or write feedback to many similar answers at once. Their interface focused on powergrading for short-answer questions from the U.S. Citizenship exam. After machine learning clustered answers, the frontend allowed teachers to read, grade, or provide feedback on similar answers simultaneously. When compared against a baseline interface, the teachers assigned grades to students substantially faster, gave more feedback to students, and developed a ``high-level view of students' understanding and misconceptions'' \cite{basuDivideAndConquer}.

OverCode \cite{overcode} took steps toward enabling powergrading in the domain of programming education. The system enabled teachers to visualize and explore the thousands of student submissions to simple exercises in an introductory programming MOOC. OverCode used static and dynamic analysis to cluster similar solutions on the basis of variable behavior, and then presented these ``stacks'' to the teacher. It was found that the system enabled teachers to more quickly understand the different strategies and misconceptions used by students. Foobaz builds upon the OverCode pipeline, using the stacks and common variables it produces as the basis for delivering feedback on variable names at scale.

Foobaz presents a significant departure from OverCode. The Foobaz system uses the OverCode program analysis backend to bring to the fore what OverCode intentionally hid: variable names. In order to create the new user interface, we developed a technique for visualizing the variation of names within clusters. The feedback mechanism is also distinct. OverCode helped teachers write general feedback for the entire class, while Foobaz creates personalized feedback quizzes for each student.

Two more recently published systems help teachers give programming students subjective feedback on coding style: AutoStyle \cite{autostyle} and ACES \cite{ACES}. AutoStyle is designed for automatically composing code style feedback to programming students at scale. Style, in this system, refers to the effective use of programming idioms; it does not allow for feedback on variable names, indentation, or punctuation. ACES relies on static analysis, Abstract Syntax Trees, and unsupervised learning to streamline the process of grading on style. The analysis backend recommends feedback for each new submission based on past solutions and teacher annotations. However, in its user interface, the teacher still reviews submissions one at a time, ultimately limiting its ability to scale.

% Taherkhani - not sure where to fit in
% Taherkhani et al. [2012, 2013] \cite{} categorize variables based on variable behavior. Both methods consider the sequence of values to which variables are assigned. Unlike Taherkhani et al., OverCode does not attempt to categorize variable behavior as one of a set of predefined roles.


\subsection{Variable Name Design}
Designing names for variables is an art more than a science. Donald Knuth compares a good programmer to an essayist who, ``with thesaurus in hand, chooses the names of variables carefully and explains what each variable means'' \cite{literateprogramming}. Without modifying execution, names can express to the human reader the type and purpose of an object, as well as suggest the kinds of operators used to manipulate it \cite{operands}.

The freedom that programmers have when naming classes, functions, and variables allows them to name variables poorly. At best, bad variable names are the subject of humor, i.e., ``26 Variable Names for Busy Developers: a, b, c, d, e...'' \cite{hackeronion}. Various naming conventions, like Hungarian notation, have evolved to help developers use their freedom wisely. The Google C++ Style Guide authors assert that their most important consistency rules govern naming, which are arbitrary but consistent in order to increase human readability \cite{GoogleCStyleGuide}. 

One of MIT's largest core software engineering courses (6.005) \cite{UseGoodNames6.005CodeReviewReading} specifically recommends students use verb phrases for method names and noun phrases for variable and class names. The 6.005 staff also ask that student balance the need for descriptive and meaningful names with conciseness. Finally, abbreviations are considered bad form; they can be difficult to unpack for both native and non-native English speakers.

Programmers can develop their own heuristics for good variable names through the experiential learning process of building, debugging, and sharing increasingly large programs with others and their future selves. During interviews, one professor explained an elaborate set of guidelines that she personally developed and teaches to her students that are specific to the domain in which she works, i.e., machine learning [Finale Doshi, personal communication].

\subsection{Variable Names in Classrooms}
Bad variable names can throw roadblocks into the paths of already struggling beginners. Introductory programming students will, for example, iterate over the elements of an array but name the iterator as if it is an index, and vice versa [Guttag, personal communication]. It could be an innocent mistake that lengthens debugging time or indicative of a flawed mental model. Rapid feedback on variable names may remind students why naming matters, correct their flawed mental models, and expose them to examples of teacher-endorsed naming conventions and styles \cite{ieeeRapidFeedback}.




\section{Learnersourcing Personalized Hints}

Our work builds on prior research on delivering personalized support to students. It is also informed by existing research on the pedagogical benefits of reflection and explanation.

\subsection{Personalized Support}

Several types of solutions have been deployed to help students get the personalized attention they need. These solutions span the spectrum from recruiting more teaching assistants from the ranks of previous students \cite{communityTAs} to automating hints using intelligent tutoring systems. 

Intelligent tutoring systems can provide personalized hints and other assistance to each student based on a pre-programmed student model. For example, previous systems sought to provide support through the use of adaptive scripts \cite{kumar2007tutorial}, or cues from the student’s problem-solving actions \cite{diziol}. Despite the advantage of automated support, intelligent tutoring systems often require domain experts to design and build them, making them expensive to develop. \todo{Include ``current ITSs require an exact formalization of the underlying domain knowledge
which is usually a substantial amount of work: researchers have reported 100-1000 hours
of authoring time needed for one hour of instruction [MBA03] from `Feedback Provision Strategies in Intelligent Tutoring
Systems Based on Clustered Solution Spaces'''} Furthermore, domain experts who generate these hints may also suffer from the ``curse of knowledge’’: the difficulty experts have when trying to see something from a novice’s point of view \cite{curse}. 

Unlike intelligent tutoring systems, the HelpMeOut system \cite{helpmeout} does not require a pre-programmed student model. It assists programmers during their debugging by suggesting code modifications mined from debugging performed by previous programmers. However, the suggestions lack explanations in plain language unless they are added by experts (teachers), so the limits imposed by the time, expense, and curse of knowledge of experts still apply.

Discussion forums derive their value from the content produced by the teachers and students who use them. These systems can harness the benefits of peer learning, where students can benefit from generating and receiving help from each other. However, as the system has no student model, the information is available to all students whether or not it is ultimately relevant. Students can receive personalized attention only if they post a question and receive a response. 

\subsection{Reflection and Explanation}
In this work, we aim to design opportunities for students to help others while simultaneously reflecting on their own solutions. Existing theories indicate that reflection is a critical method for triggering the transformation from conflict and doubt into clarity and coherence \cite{dewey1933}. Turning that reflection into a self-explanation further improves understanding \cite{selfexplanation}. According to Turns et al. \cite{asee}, the absence of reflection in traditional engineering education is a significant shortcoming. 

Novices may become confused if asked to reflect on their solution or compare it to a fellow student’s solution; this is not necessarily bad for learning outcomes. Piaget theorized that cognitive disequilibrium, experienced as confusion, could trigger learning due to the creation or restructuring of knowledge schema \cite{disequilibrium}. D’Mello et al. maintain that confusion can be productive, as long as it is both appropriately injected and resolved \cite{productiveconfusion}. 

Similarly, reflecting on a peer’s conceptual development or alternative solution may bring about cognitive conflict that prompts reevaluation of the student’s own beliefs and understanding \cite{kavanagh}. As such, peer instruction \cite{mazur} and peer assessment \cite{peerassessment} have not only been integrated into many classroom activities, but have also formed the basis of several online systems for peer-learning. For example, Talkabout organizes students into discussion groups based on characteristics such as gender or geographic balance \cite{talkabout}.

Recent work on learnersourcing proposes that learners can collectively generate educational content for future learners while engaging in a meaningful learning experience themselves \cite{kim2013learnersourcing,weir2015,mitros2015}. For example, Crowdy enables people to annotate how-to videos while simultaneously learning from the video \cite{weir2015}. Beyond existing work, we investigate alternatives for what support students should be prompted to provide, based on their own work as well as the needs of their peers. We also explore several ways to trigger productive reflection as a byproduct of hint creation, by prompting students to either self-reflect or compare their own solutions to those produced by peers. 


\begin{comment}
    Recent years have witnessed an explosive growth in the development of novel high--throughput technologies
    for probing cellular function.  These technologies have produced a flood of data, most notably from DNA and
    protein sequencing efforts, but also from metabolic phenotyping, gene expression, proteomics, and physiological
    experiments (cf.\ Figure~\vref{fig:probingCellularFunction}).  The volume of these data is so staggering as to
    defy traditional publication, leading instead to direct submission into various, increasingly bloated,
    databases such as \genbank~(cf.\ Figure~\vref{fig:genbankGrowth}).  These databases are extremely diverse and numerous, ranging from species--specific gene expression databases to databases
    for gene sequences and protein structures.

    This diversity presents a pressing question, which is central to the
    field of systems biology: How do we integrate and analyze
    these databases in a holistic and biologically meaningful manner?  Unfortunately, efforts
    to date have produced a fragmented and confusing landscape of largely database--specific analysis tools.  These overly
    specific tools are ill suited to our needs as we begin
    to pose more probing questions of biology, particularly with regard to the systemic interdependence of cells.  Instead,
    we require \emph{generic} techniques, applicable to a diverse array of data types, for elucidating biological \emph{associations}.
    In this context, ``associations'' are essentially very rudimentary statements like ``gene A is related to gene B'' or
    ``motif C is involved in apoptosis''.  Such simple statements reflect the naivety of our current understanding of
    complex biological systems; however, they can be very powerful as indicators of topics worthy of further, more specific,
    study.

    In this proposal, I will focus on \emph{syntactic pattern discovery} as a generic tool for revealing such associations
    in diverse data sets.  Broadly speaking, syntactic pattern discovery techniques are used for finding patterns in
    data that can be projected into streams of primitives.  This problem is roughly analogous to finding the syntax
    rules in a book of an unknown language.  In the context of DNA microarray data, this book may consist of expression
    levels of thousands of genes.  Or, in clinical data, this book may contain answers to lifestyle questions asked
    of cancer patients.  In the latter case, we would be looking for associations like ``people
    who \emph{smoke} and are \emph{over 50} are likely to have \emph{prostate} cancer''.  This pattern is conceptually no different
    than a functional motif in the context of protein sequences.  For example, we can use syntactic pattern discovery to find
    associations such as ``the motif `\texttt{KT..GA..R}' is associated with dehydrogenase enzymes.''

    Techniques for finding associations in data, such as syntactic pattern discovery, are extremely useful for pinpointing interesting phenomena,
    but do not seek to explain them.  As such, these techniques are sufficiently generic to find wide applicability in the diverse and copious
    databases that have become available.  The continued growth of these databases will ensure
    that techniques like syntactic pattern discovery will play a pivotal role in directing the course of biological research in the coming years.



        \begin{figure}[tbph!]
        \centering
            %\include{Body/Images-chap1/fig_cell}
        \caption[Probing Cellular Function]{Probing Cellular Function.  This figure shows just of a few of the diverse set of
        techniques available to researchers for probing the function of cells.  Due to the high--throughput nature of
        these techniques, data bases are rapidly becoming prohibitively volumous.}
        \label{fig:probingCellularFunction}
        \end{figure}

        \begin{figure}[ptbh]
        \centering
            %\include{Body/Images-chap1/fig_genbank}
        \caption[Exponential growth of \genbank ]{Exponential growth of \genbank .  \genbank~has increased by
                and order of magnitude nearly every three years and has grown to include approximately $10^{10}$ DNA base--pairs
                and $10^7$ sequences from over 55,000 different species~\cite{benson2000genbank}.}
        \label{fig:genbankGrowth}
        \end{figure}

\section{General Objectives and Specific Aims}

    \subsection{General Objectives}
        The central theme of the work proposed here is the use of
        syntactic pattern discovery techniques to find novel associations
        in biological systems.  In particular, we will focus on protein
        sequence, DNA sequence, gene expression, and physiological data
        being generated by both our own laboratory and by collaborators.
        We will use these data to solve a number of problems of both
        practical and scientific interest to the life sciences community,
        which are described in Section~\vref{subsection:specificAims}.

        The manner in which we will approach these problems, and the
        types of methods that we will employ are probably best explained
        through a few short examples.  Below, I will describe two very
        simple problems in which syntactic pattern discovery proves to be
        a powerful tool for elucidating interesting biological phenomena.
        It is important to note that, despite the qualitative difference
        between these two problems, they can be analyzed using the same
        generic framework.

        \subsubsection{A Few Simple Examples}
            \begin{description}
            \item[Breast Cancer Association Discovery:]
                Table~\vref{table:breastCancerData} shows
                a representative sampling of a larger data set containing tumor statistics from breast cancer patients.  Given data
                such as these, we are interested in finding associations between the different columns which may have some biological meaning.
                For example, in this data set, we find that patients who have rough and low--density tumors are more likely to develop recurrent
                cancers.  However, these data do not indicate that merely having a rough \emph{or} a low--density tumor is enough to predispose
                a patient to recurrence.  This suggests that there is some association between the two factors that may have
                a biological underpinning, for example a mechanism of malignant cell growth.  Such associations can be strong indicators
                of areas worthy of further study.

                \begin{table}[!hbtp]
                \centering
                \caption[Association discovery in cancer patient data]{Association discovery in cancer patient data.  These data are just a
                small fraction a complete clinical study of breast cancer patients by Mangasarian \etal~\cite{mangasarian1995breast} available in
                the University of California Irvine Machine Learning Repository~\cite{uci1998ucirepository}.  The first column indicates whether
                or not the patient's cancer returned.  All other columns contain an integer, 1--10, indicating the percentile in which the
                patient lies for a specific metric.  For instance, the first patient had a tumor that was in the $20^{\textrm{th}}$ percentile
                in area.  Using syntactic pattern discovery, we can find a number of associations in the full data set.  For example, patients
                who have very rough and low--density tumors ($10^{\textrm{th}}$ percentiles in both cases) are likely to have recurrent cancers.  However,
                these factors are not necessarily important individually; there is an association between the two.}\label{table:breastCancerData}
                    \include{Body/Images-chap1/table_breast}
                \end{table}

            \item[Gene Local Homology Discovery:]Table~\vref{table:arabGenes} shows a few genes from the Arabodopsis genome.  Given sequence data, we are typically
                interested in identifying areas of similarity between two sequences or finding functional motifs.  A rigorous example
                of either of these problems requires more development.  But, in
                this very simple case, we may be interested in finding long subsequences of these gene which are present in all three.  Other
                than the poly--A tails, there are only a few such subsequences, for example \texttt{CCACGCGTCCGAAAA}.  In some problems, we
                may use such conserved regions to deduce an evolutionary history of these genes or try to correlate these regions with a
                certain function. 

                \begin{table}[!hbtp]
                \centering
                \caption[Gene sequences from Arabodopsis]{Genes sequences from Arabodopsis.  These very small genes are only a
                few hundred bases long, whereas a typical gene can be many kilo--bases in length.  At first glance, only the poly--A tails
                stand out as areas of local homology; however other long strings, such as \texttt{CCACGCGTCCGAAAA} appear in all three sequences.
                In databases that contain literally billions of these sequences, we would use syntactic pattern discovery to find these
                kinds of patterns.}\label{table:arabGenes}
                    \include{Body/Images-chap1/table_gene_similarity}
                \end{table}
                \end{description}

    \subsection{Specific Aims}\label{subsection:specificAims}
        The work proposed here has two specific goals:
        \begin{enumerate}
            \item   To use syntactic pattern discovery in the space of protein and DNA sequences.  Specifically, we will
                    develop a model of protein molecular evolution based on large--scale syntactic pattern discovery in
                    the set of natural proteins.  Also, we will develop a method for predicting cDNA oligonucleotide
                    hybridization kinetics and selecting optimal probes for DNA microarrays using syntactic pattern
                    discovery.
            \item   To elucidate novel biological associations in expression and physiological data.  Using data that
                    is being generated in our own laboratory and by collaborators, we will apply syntactic pattern discovery
                    to create leads for further, in--depth, research.  We will pay particular attention to a few model systems
                    including diabetes and various cancers.
        \end{enumerate}




\section{Tools for Pattern Discovery}
    \subsection{Introduction}\label{pdIntro}
        The general pattern discovery problem is as follows: given a data
        set, find the underlying patterns in the data.  Depending on what
        we mean by a ``pattern'' and what kinds of data we are given,
        this problem can take on many forms.  Figure~\vref{fig:gaussian}
	shows a very simple pattern discovery problem
        in which we use a clustering technique to find patterns.

        It is important to distinguish between pattern \emph{discovery} and pattern \emph{recognition}.  In pattern
        recognition we are given patterns and we search for occurrences of these patterns in a data set.  This, relatively
        simple, task is closely associated with the concept of classification in which we partition the data set into groups
        depending on which patterns the data match.  Pattern discovery is essentially the inverse of pattern
        recognition and is a much more difficult task: given the data, find the groups.



        From Figure~\vref{fig:gaussian}, it is obvious that there are many different ways to answer the
        pattern discovery problem depending on our predefined notion of what a pattern should look like.  For example,
        we may say that a pattern is a collection of dots that lie within a circle of a certain radius.  Or, we could
        say that a pattern is a set dots in which each dot is no more than a certain distance away from every other dot.
        Essentially, we need to define the
        pattern \emph{class} for which we will look.  This is an fundamental tenet of the general pattern discovery problem.

        In this work, we will deal exclusively with data that consists of sequences of characters; however, the underlying
        principles of pattern discovery remain unchanged.  For example, given two character strings, ``\texttt{thesiscommittee}''
        and ``\texttt{iscommitted}''\footnote{As in, the ``\texttt{thesiscommittee}'' ``\texttt{iscommitted}'' to getting me out in a minimal amount of time.}, what are the patterns is these strings?  One answer is that they both contain letters from the
        set \{\texttt{t,h,e,s,c,o,m,i,t}\}.  Another answer is that they both contain the substring ``\texttt{iscommitte}'', and yet
        another answer is that they both contain the substring ``\texttt{commit}''.  Depending on how exactly we choose the
        pattern class in which we are interested, all of these answers are correct.  This concept will be discussed further in
        Section~\vref{pdIntro}.

        In what follows, I will briefly highlight a number of pattern discovery tools, distinguishing between two types: decision--theoretic
        pattern discovery tools and tools for syntactic pattern discovery.  Then, I will give a review of the \teiresias~algorithm,
        which will be the pattern discovery tool of choice for the work proposed herein.

    \subsection{Decision--Theoretic Pattern Discovery }
        Decision--theoretic pattern discovery refers to a wide variety of techniques that operate, in general, on data
        which is in vector form.  These techniques can be divided into three main areas depending on how pattern classes
        are determined: decision functions, distance functions, and likelihood functions~\cite{tou1974pattern}~\cite{duda1973pattern}.  The reader is probably most
        familiar with decision--theoretic pattern discovery techniques such linear discriminant analysis, K--means clustering,
        and hierarchical clustering.  Each of these are appropriate for elucidating different pattern classes and typically
        find most use on different forms of data.

        The most important, distinguishing, characteristic of decision--theoretic pattern discovery algorithms is that
        they are used for finding patterns in vector data.  Thus, many of these techniques are widely applicable and
        have been thoroughly exploited in the biology literature.  For example, various clustering techniques have been
        very useful for analyzing gene expression experiments.  However, these same techniques find very little use
        in sequence analysis.


    \subsection{Syntactic Pattern Discovery }
        \subsubsection{Introduction}
            Syntactic pattern discovery is the task of finding patterns in streams of primitives.  Usually, these streams
            are strings of characters, for example amino acids in a protein sequence.  However, any data set that can be
            written in terms of primitives is well--suited for syntactic pattern discovery.  For example, given the expression
            level of a gene, we may call it ``up--regulated'', ``down--regulated'', or ``no change''.  So, a sequence of
            gene expression data may be written as ``\texttt{U D N U U U D}'', where ``\texttt{U}'', ``\texttt{D}'',
            and ``\texttt{N}'' are the primitives that make up our data stream.  Given a few such streams like
            ``\texttt{U D N U U U D}'', ``\texttt{D D N U D U U}'', and ``\texttt{D D U U N U N}'',
            we would use syntactic pattern to discovery that genes 2, 4, and 6 are always expressed as ``\texttt{D}'', ``\texttt{U}'',
            and ``\texttt{U}'' together, respectively --- they are co--regulated.

            The underlying framework of syntactic pattern discovery can be traced back to Chomsky's work on syntax theory~\cite{chomsky1957syntactic,chomsky1965aspects,chomsky1956three}.
            Chomsky's work is the basis for much of formal language theory and computational linguistics.  However, in general,
            most work in these areas has been syntactic pattern \emph{recognition}, and has focused on machine learning techniques
            for computer understanding of natural languages.  Only recently have these pattern recognition techniques been applied to problems of
            interest to biologists~\cite{searls1997linguistic,searls2001reading,searls1992thecomputational}, such as gene finding.

            Though descended from formal language theory, syntactic pattern discovery has evolved independently
            in the field of computational biology through efforts in the area of sequence analysis.  In general, the
            goal of syntactic pattern discovery in biological sequence data is to find patterns or motifs that may
            have biological significance.  For example, using the \teiresias~\cite{rigoutsos1998combinatorial} syntactic pattern
            discovery algorithm, Rigoutsos \etal~\cite{rigoutsos1999dictionary} searched the SWISS--PROT/TrEMBL~\cite{bairoch2000swiss-prot} protein sequence database
            for important patterns and found many motifs such as ``\texttt{G..G.GK[ST]TL}'' (see Section~\ref{pdIntro} for
            an description of this notation), which is the well--known ATP/GTP binding P--loop.

            In what follows, I will briefly introduce some common nomenclature and notation used in the syntactic pattern
            discovery literature.  Then, I will describe the different types of syntactic pattern discovery algorithms
            that have been developed, focusing in particular on the \teiresias~algorithm.


        \subsubsection{Definitions}\label{synPdIntro}

            We define an alphabet of all characters to be $\Sigma$, where $\Sigma$ is usually either the alphabet of all possible
            amino acids or the alphabet of nucleotides.  Then a sequence $s_i$ is a string consisting of zero or more of the
            letters from $\Sigma$.  We denote this by $s_i\in\Sigma^{\ast}$, which means ``string $s_i$ is an element of the
            set of strings consisting of zero or more letters'', where the ``$\ast$'' denotes ``zero or more''.

            The generic problem of pattern discovery is, given a set
            of sequences $S = \{ s_1,s_2,\ldots,s_n\}$, and an integer
            $K$, find all patterns which occur at least $K$ times
            in $S$.  We call this set of patterns $\mathcal{C}$, where
            $\mathcal{C}=\{ p_1,p_2,\ldots,p_m\}$ and $p_i$ is a pattern
            in $\mathcal{C}$.  Each pattern $p_i$ defines a language
            $\mathcal{L}\paren{p_i}$ which is the set of all strings that
            can be derived from $p_i$.  A sequence ``matches'' $p_i$ if it
            contains a substring that belongs to $\mathcal{L}\paren{p_i}$.
            For example, if $p_i=$``\texttt{KYLE}'', then the sequences
            ``\texttt{GOTOBEDKYLE}'', ``\texttt{WRITEYOURPROPOSALKYLE}'',
            and ``\texttt{GOODBOYKYLEWRITE}'' all match $p_i$.

            As mentioned in Section~\vref{subsection introduction tools
            for pat disc}, to completely determine the pattern discovery
            problem, we have to specify the pattern class in which we are
            interested.  This pattern class determines the form of each
            $p_i$ that we find.  Below, a few pattern classes, commonly
            used in biological sequence analysis,  are enumerated in
            order of increasing complexity~\cite{floratos1999pattern}:

            \begin{itemize}
                \item   $p_i\in\Sigma^{\ast}$:  This is the class
                        of all ``solid'' patterns, for example
                        ``\texttt{KAGTPT}'' and ``\texttt{TAGCGGGAT}''.

                \item   $p_i\in(\Sigma\cup\{.\})^{\ast}$:  This is the
                        class of all patterns that can have ``wildcard''
                        positions, which are denoted by ``.'', for example
                        ``\texttt{K.G.PT}'' and ``\texttt{TA...GGAT}''.
                        The wildcard means that any character from the
                        alphabet will suffice in that position.

                \item   $p_i\in(\Sigma\cup R)^{\ast}$: This
                        is the class of all patterns that can
                        have ``bracketed'' expressions, for
                        example ``\texttt{K[ADG]G[KQ]PT}'' and
                        ``\texttt{TA[GA][TC]GGAT}''.  The bracketed
                        expression ``[\texttt{TC}]'' means that either
                        ``\texttt{T}'' or ``\texttt{C}'' will suffice in
                        that position.  In this notation, $R$ represents
                        the set of characters in the brackets, for example
                        $R=\{\texttt{TC}\}$ or $R=\{\texttt{GA}\}$.

                \item   $p_i\in(\Sigma\cup X)^{\ast}$:  This is the
                        class of ``flexible'' patterns.  For example
                        ``\texttt{Kx(1,3)Gx(2,5)PT}'', where ``x(2,5)''
                        means that anywhere between two and five wildcards
                        can exist at that position, that is x(2,5)
                        can be any one of $\{..,...,....,......\}$.

            \end{itemize}
            In general, the more complex these patterns are, the more expressive the languages will be that we find.  However,
            with increasing complexity, the computational difficulty of the pattern discovery problem increases
            drastically~\cite{maier1978complexity,garey1979computers}.

            There are three characteristics that distinguish between various syntactic pattern discovery algorithms, which are as
            follows:
            \begin{enumerate}
                \item   The pattern class complexity.
                \item   The completeness of the returned pattern set.  That is, does the algorithm return \emph{all}
                        patterns present in the input sequences?
                \item   The pattern \emph{maximality}.  For instance, in the two strings ``\texttt{KYLEJ}'' and ``\texttt{KYLEL}'',
                        the pattern ``\texttt{KYLE}'' is maximal, whereas ``\texttt{KYL}'' is not, because we could
                        add an ``\texttt{E}'' without decreasing the number of times it occurs.
            \end{enumerate}

        \subsubsection{Syntactic Pattern Discovery Algorithms}\label{subsubsection: syntax pattern disc algos}

            There are two principal classes of algorithms used for syntactic pattern discovery in biological sequences:
            sequence driven and pattern driven algorithms~\cite{brazma1998pattern}.  Sequence driven algorithms align a set of strings,
            which are presumed to be similar, producing a ``consensus sequence''~\cite{gusfield1997algorithms}.   This consensus sequence is, in essence,
            an average of the input sequences and is used to capture patterns in the input sequences (cf.\ Figure~\vref{fig:consensusSequence})\cite{neville1997enumerating}.  The most well--known sequence driven syntactic
            pattern discovery algorithm is BLOCKS~\cite{henikoff1991automated}, others include EMOTIF~\cite{neville1997enumerating} and
            algorithms by Martinez~\cite{martinez1988flexible}, Smith \etal~\cite{smith1990automatic}, Vingron \etal~\cite{vingron1991motif},
            Shinohara~\cite{shinohara1982poly}, Nix~\cite{nix1984editing}, Roytberg~\cite{roytberg1992search}, Schuler~\cite{schuler1991workbench},
            Brodsky~\cite{brodsky1992mathematical}, and Clift~\cite{clift1986sequence}.

            Sequence driven syntactic pattern discovery algorithms, because they allow for insertions and deletions, are capable of finding
            relatively complex flexible patterns~\cite{floratos1999pattern}.  Also, by using common sequence alignment tools, which use heuristical short--cuts, they typically
            have low run--time costs.  However, because a number of input sequences are required to produce a single consensus sequence, these pattern discovery
            algorithms typically are not guaranteed to find \emph{all} the patterns.  Additionally, sequence driven algorithms are not
            guaranteed to be maximal.  As such, these algorithms are best--suited to situations in which the input sequences are known to be
            globally similar and finding all of the patterns is not important.

            \begin{figure}[tbph!]
            \centering
                %\include{fig/fig_consensus}
            \caption[Alignment with a consensus sequence]{Alignment with a consensus sequence.  Alignment--based syntactic pattern
            discovery algorithms use the consensus sequence to extract patterns from the input sequences.  In this figure, characters
            that are conserved in all sequences are capitalized.  Note that the consensus
            sequence ``{\ttfamily abd\textbf{F}l\textbf{R}-\textbf{P}abd\textbf{F}l\textbf{T}-\textbf{Q}}'' shows the pattern
            ``{\ttfamily ...F.R.P...F.T.Q}''.}
            \label{fig:consensusSequence}
            \end{figure}

            Pattern driven syntactic pattern discovery algorithms work by first enumerating possible patterns and then checking
            for the patterns in the sequence set~\cite{brazma1998pattern}.  In contrast to sequence driven algorithms, it is possible in this case to
            guarantee the completeness of the set of returned patterns.  However, this guarantee comes at the price of increased
            time and space complexity.  That is, the set of all possible patterns is very large and can take a large space
            to enumerate and a long time to search through\footnote{In fact, for arbitrarily complex patterns, this problem is
            NP--hard~\cite{garey1979computers,maier1978complexity}}.  As such, most pattern driven pattern discovery algorithms use
            heuristics to limit the space of patterns that are searched.  Common pattern driven algorithms include those by
            Queen \etal~\cite{queen1982improvements}, Waterman \etal~\cite{waterman1984pattern}, Staden~\cite{staden1989methods},
            Wolferstetter \etal~\cite{wolferstetter1996identification}, Smith \etal~\cite{smith1990finding}, Sagot \etal~\cite{sagot1997multiple},
            Suyama \etal~\cite{suyama1995searching}, Neuwald \etal~\cite{neuwald1994detecting}, Jonassen \etal~\cite{jonassen1995finding} and
            Floratos and Rigoutsos~\cite{floratos1999pattern,rigoutsos1998combinatorial}.



    \section{Papers to consider}
	\citet{yaffe2001motif}\\
	\citet{scott2004predicting}\\
	\citet{buck2005networks}\\
	\citet{robinson2004simple}\\
	\citet{yang2004predicting}\\
	\citet{pearson1990rapid}\\
	\citet{pearson1991searching}\\
	\citet{maier1978complexity}\\
	\citet{salgado2004regulondb}\\
	\citet{prakash2005statistics}\\
	\citet{tompa2005assessing}\\
	\citet{venter2001sequence}\\
	\citet{lander2001initial}\\
	\citet{browning2004regulation}\\
	\citet{zhang2000greedy}\\
	\citet{wheeler2005database}\\
	\citet{styczynski2004extension}\\
	\citet{eddy1998profile}\\
	\citet{garey1979computers}\\
	\citet{stormo1982use}\\
	\citet{stormo1982use}\\
	\citet{staden1984computer}\\
	\citet{duda2000pattern}\\
	\citet{heger2003sensitive}\\
	\citet{tomita1989optimal}\\
	\citet{zaki2000scalable}\\
	\citet{murthy2003rnabase}\\
	\citet{eskin2002finding}\\
	\citet{keich2002finding}\\
	\citet{keich2002subtle}\\
	\citet{price2003finding}\\
	\citet{thompson1994clustal}\\
	\citet{kabsch1983dictionary}\\
	\citet{williams2003multiple}\\
	\citet{bajic2004promoter}\\
	\citet{eddy2004how}\\
	\citet{bateman2004pfam}\\
	\citet{bairoch2000enzyme}\\
	\citet{marchler2003cdd}\\
	\citet{kel2004recognition}\\
	\citet{sinha2003discriminative}\\
	\citet{sinha2003ymf}\\
	\citet{lee2003generating}\\
	\citet{haverty2004cisml}\\
	\citet{fu2004motifviz}\\
	\citet{frith2004detection}\\
	\citet{poluliakh2003melina}\\
	\citet{frith2004finding}\\
	\citet{jonassen2002structure}\\
	\citet{stormo2000dna}\\
	\citet{lepre2004genes}\\
	\citet{mancheron2003pattern}\\
	\citet{hofacker2004alignment}\\
	\citet{wingender2000transfac}\\
	\citet{sinha2000statistical}\\
	\citet{sinha2003performance}\\
	\citet{pevzner2000combinatorial}\\
	\citet{duda1973pattern}\\
	\citet{tou1974pattern}\\
	\citet{hertz1999identifying}\\
	\citet{henikoff1995automated}\\
	\citet{lawrence1993detecting}\\
	\citet{bailey1994fitting}\\
	\citet{jonassen1995finding}\\
	\citet{agrawal1995mining}\\
	\citet{altschul1990basic}\\
	\citet{altschul1991amino}\\
	\citet{altschul1996local}\\
	\citet{altschul1997gapped}\\
	\citet{bradley2002trilogy}\\
	\citet{brazma1997approaches}\\
	\citet{brazma1998predicting}\\
	\citet{buhler2001finding}\\
	\citet{burge2000prediction}\\
	\citet{chomsky1956three}\\
	\citet{dayhoff1979amodel}\\
	\citet{dehal2002thedraft}\\
	\citet{dhaeseleer2000genetic}\\
	\citet{dsouza1997searching}\\
	\citet{floratos1999pattern}\\
	\citet{guhathakurta2002identifying}\\
	\citet{henikoff1991automated}\\
	\citet{henikoff1992aminoacid}\\
	\citet{hughes2000computational}\\
	\citet{jonassen1996methods}\\
	\citet{jurafsky2000speech}\\
	\citet{karlin1990methods}\\
	\citet{kim2003bioinformatic}\\
	\citet{levy1998xlandscape}\\
	\citet{li2001selection}\\
	\citet{lin2002finding}\\
	\citet{liu2001functional}\\
	\citet{manber1993suffix}\\
	\citet{parida1998musca}\\
	\citet{pearson1998improved}\\
	\citet{pedersen1999thebiology}\\
	\citet{pesole2000patsearch}\\
	\citet{rigoutsos1998combinatorial}\\
	\citet{rigoutsos1999dictionary}\\
	\citet{rigoutsos2000theemergence}\\
	\citet{searls1992thecomputational}\\
	\citet{searls2002language}\\
	\citet{smith1981identification}\\
	\citet{waterman1984efficient}\\
	\citet{wootton1993statistics}\\
	\citet{yaffe2001motif}\\


\end{comment}
