% $Log: abstract.tex,v $
% Revision 1.1  93/05/14  14:56:25  starflt
% Initial revision
%
% Revision 1.1  90/05/04  10:41:01  lwvanels
% Initial revision
%
%
%% The text of your abstract and nothing else (other than comments) goes here.
%% It will be single-spaced and the rest of the text that is supposed to go on
%% the abstract page will be generated by the abstractpage environment.  This
%% file should be \input (not \include 'd) from cover.tex.
Massive programming courses produce a massive collection of student solutions for each programming exercise. The solutions to any particular problem vary along many dimensions, including bugs, naming, syntax, and semantics. The distribution of solutions along these dimensions reflect students' prior knowledge, the teacher's course curriculum and explanations so far, and misconceptions common to all programming students.

Personal tutors can respond to an individual tutee's solution, and only draw on their personal recollection of previously observed solution variation. Teachers teaching larger groups of students can directly observe that a significant fraction of students are struggling with a particular concept or implementation, and respond appropriately with rapid contextual feedback. They can also pick out particular student solutions as examples to illustrate different concepts or ways of solving a problem, rather than solely relying on their own creativity to composing these examples from scratch. 

When scaling up to massive programming courses, it becomes painful or prohibitively exhausting to engage with student solutions this way. It is also an opportunity: teachers have access to a comparatively dense sampling of the distribution over student solution bugs, naming, syntax, and semantics. 

The systems in this thesis help teachers take advantage of the massive collections of solutions, enabling either (1) the same teaching practices that were previously only tractable in smaller courses or (2) new practices that are only possible when a massive collection of student solutions are available. The common empowering mechanism is clustering and visualizing solution variation for human understanding. Using these systems, teachers can gain insights into student design choices, detect autograder failures, process solutions that deserve partial credit, use targeted learnersourcing to collect hints for other students, and give personalized style feedback at scale.
