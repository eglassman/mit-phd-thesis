I am not a professional software developer, but learning how to write programs in middle school was one of the most empowering skills my father could ever have taught me. I did not need access to chemicals or heavy machinery. I only needed a computer and occasionally an internet connection. I acquired datasets and wrote programs to extract interesting patterns in them. It was and still is a creative outlet.

More recently, the value of learning how to code, or at least how to think computationally, has gained national attention. Last September, New York City Mayor Bill de Blasio announced that all public schools in NYC will be required to offer computer science to all students by 2025. In January of this year, the White House released its Computer Science for All initiative, "offering every student the hands--on computer science and math classes that make them job--ready on day one" (President Obama, 2016 State of the Union Address).

The economic value and employment prospects associated with knowing how to program, the subsiding stigma of being a computer "geek" or "nerd," and the production and popularity of movies about programmers may be responsible for driving up enrollment in computer science classes to unprecedented levels. Hundreds or thousands of students enroll in programming courses at schools like MIT, Stanford, Berkeley and the University of Washington.

One-on-one tutoring is considered a gold standard in education, and programming education is likely no exception. However, most students are not going to receive that kind of personalized instruction. We, as a computer science community, may not be able to offer one-on-one tutoring at a massive scale, but can we create systems that enhance the teacher and student experiences in massive classrooms in ways that would never have been possible in one-on-one tutoring? This thesis is one particular approach to answering that question. %are not even possible in just a one-on-one environment?% How can we best teach the next generation of students how to code when there are so many of them?
