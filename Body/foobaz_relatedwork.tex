\section{Foobaz}
Foobaz builds upon past systems for enabling grading at scale, particularly in the context of teaching students how to program well. We also provide background on the principles of good variable naming.

\todo{add "What's in a name?" and binkley2011improving papers and The Java Programmer’s Phrase Book and Debugging Method Names; When a function is uncommented, a human reader’s program comprehension may depend almost entirely on variable names (Lawrie et al., ‘06)--talk about its experiment conclusions.}

\subsection{User Interfaces for Grading at Scale}
The powergrading paradigm \cite{basupowergrading} enables teachers to assign grades or write feedback to many similar answers at once. Their interface focused on powergrading for short-answer questions from the U.S. Citizenship exam. After machine learning clustered answers, the frontend allowed teachers to read, grade, or provide feedback on similar answers simultaneously. When compared against a baseline interface, the teachers assigned grades to students substantially faster, gave more feedback to students, and developed a ``high-level view of students' understanding and misconceptions'' \cite{basuDivideAndConquer}.

OverCode \cite{overcode} took steps toward enabling powergrading in the domain of programming education. The system enabled teachers to visualize and explore the thousands of student submissions to simple exercises in an introductory programming MOOC. OverCode used static and dynamic analysis to cluster similar solutions on the basis of variable behavior, and then presented these ``stacks'' to the teacher. It was found that the system enabled teachers to more quickly understand the different strategies and misconceptions used by students. Foobaz builds upon the OverCode pipeline, using the stacks and common variables it produces as the basis for delivering feedback on variable names at scale.

Foobaz presents a significant departure from OverCode. The Foobaz system uses the OverCode program analysis backend to bring to the fore what OverCode intentionally hid: variable names. In order to create the new user interface, we developed a technique for visualizing the variation of names within clusters. The feedback mechanism is also distinct. OverCode helped teachers write general feedback for the entire class, while Foobaz creates personalized feedback quizzes for each student.

Two more recently published systems help teachers give programming students subjective feedback on coding style: AutoStyle \cite{autostyle} and ACES \cite{ACES}. AutoStyle is designed for automatically composing code style feedback to programming students at scale. Style, in this system, refers to the effective use of programming idioms; it does not allow for feedback on variable names, indentation, or punctuation. ACES relies on static analysis, Abstract Syntax Trees, and unsupervised learning to streamline the process of grading on style. The analysis backend recommends feedback for each new submission based on past solutions and teacher annotations. However, in its user interface, the teacher still reviews submissions one at a time, ultimately limiting its ability to scale.

% Taherkhani - not sure where to fit in
% Taherkhani et al. [2012, 2013] \cite{} categorize variables based on variable behavior. Both methods consider the sequence of values to which variables are assigned. Unlike Taherkhani et al., OverCode does not attempt to categorize variable behavior as one of a set of predefined roles.


\subsection{Variable Name Design}
Designing names for variables is an art more than a science. Donald Knuth compares a good programmer to an essayist who, ``with thesaurus in hand, chooses the names of variables carefully and explains what each variable means'' \cite{literateprogramming}. Without modifying execution, names can express to the human reader the type and purpose of an object, as well as suggest the kinds of operators used to manipulate it \cite{operands}.

The freedom that programmers have when naming classes, functions, and variables allows them to name variables poorly. At best, bad variable names are the subject of humor, i.e., ``26 Variable Names for Busy Developers: a, b, c, d, e...'' \cite{hackeronion}. Various naming conventions, like Hungarian notation, have evolved to help developers use their freedom wisely. The Google C++ Style Guide authors assert that their most important consistency rules govern naming, which are arbitrary but consistent in order to increase human readability \cite{GoogleCStyleGuide}. 

One of MIT's largest core software engineering courses (6.005) \cite{UseGoodNames6.005CodeReviewReading} specifically recommends students use verb phrases for method names and noun phrases for variable and class names. The 6.005 staff also ask that student balance the need for descriptive and meaningful names with conciseness. Finally, abbreviations are considered bad form; they can be difficult to unpack for both native and non-native English speakers.

Programmers can develop their own heuristics for good variable names through the experiential learning process of building, debugging, and sharing increasingly large programs with others and their future selves. During interviews, one professor explained an elaborate set of guidelines that she personally developed and teaches to her students that are specific to the domain in which she works, i.e., machine learning [Finale Doshi, personal communication].

\subsection{Variable Names in Classrooms}
Bad variable names can throw roadblocks into the paths of already struggling beginners. Introductory programming students will, for example, iterate over the elements of an array but name the iterator as if it is an index, and vice versa [Guttag, personal communication]. It could be an innocent mistake that lengthens debugging time or indicative of a flawed mental model. Rapid feedback on variable names may remind students why naming matters, correct their flawed mental models, and expose them to examples of teacher-endorsed naming conventions and styles \cite{ieeeRapidFeedback}.



