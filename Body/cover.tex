% -*-latex-*-
% $Log: cover.tex,v $
% Revision 1.7  2001/02/08 18:53:16  boojum
% changed some \newpages to \cleardoublepages
%
% Revision 1.6  1999/10/21 14:49:31  boojum
% changed comment referring to documentstyle
%
% Revision 1.5  1999/10/21 14:39:04  boojum
% *** empty log message ***
%
% Revision 1.4  1997/04/18  17:54:10  othomas
% added page numbers on abstract and cover, and made 1 abstract
% page the default rather than 2.  (anne hunter tells me this
% is the new institute standard.)
%
% Revision 1.4  1997/04/18  17:54:10  othomas
% added page numbers on abstract and cover, and made 1 abstract
% page the default rather than 2.  (anne hunter tells me this
% is the new institute standard.)
%
% Revision 1.3  93/05/17  17:06:29  starflt
% Added acknowledgements section (suggested by tompalka)
% 
% Revision 1.2  92/04/22  13:13:13  epeisach
% Fixes for 1991 course 6 requirements
% Phrase "and to grant others the right to do so" has been added to 
% permission clause
% Second copy of abstract is not counted as separate pages so numbering works
% out
% 
% Revision 1.1  92/04/22  13:08:20  epeisach
\addcontentsline{toc}{chapter}{Cover page}
\title{Clustering and Visualizing Solution Variation in Massive Programming Classes}
\author{Elena L. Glassman}
\department{Department of Electrical Engineering \\ and Computer Science}
% If the thesis is for two degrees simultaneously, list them both
% separated by \and like this:
 \degree{Doctor of Philosophy}
%\degree{Bachelor of Science in Computer Science and Engineering}
\degreemonth{August}
\degreeyear{2016}
\thesisdate{August 7, 2016}

%% By default, the thesis will be copyrighted to MIT.  If you need to copyright
%% the thesis to yourself, just specify the `vi' documentclass option.  If for
%% some reason you want to exactly specify the copyright notice text, you can
%% use the \copyrightnoticetext command.  
\copyrightnoticetext{\copyright ~Elena L. Glassman, 2016.}

\begin{comment}
\copyrightnoticetext{\copyright ~Elena L. Glassman, 2016. \\
{\footnotesize
This work is licensed under the Creative
Commons Attribution-NonCommercial 2.5
License. To view a copy of this license, visit
\url{http://creativecommons.org/licenses/by-nc/2.5/} or send a
letter to
Creative Commons
543 Howard Street, 5th Floor,
San Francisco, California, 94105, USA.
}}
\end{comment}

% If there is more than one supervisor, use the \supervisor command
% once for each.
\supervisor{Robert C. Miller}{Professor of Electrical Engineering \\ and Computer Science}

% This is the department committee chairman, not the thesis committee
% chairman.  You should replace this with your Department's Committee
% Chairman.
%\chairman{Daniel Blankschtein}{Chairman, Department Committee on Graduate Students}
\chairman{------------------}{Chairman, Department Committee on Graduate Students}

% Make the titlepage based on the above information.  If you need
% something special and can't use the standard form, you can specify
% the exact text of the titlepage yourself.  Put it in a titlepage
% environment and leave blank lines where you want vertical space.
% The spaces will be adjusted to fill the entire page.  The dotted
% lines for the signatures are made with the \signature command.
\maketitle

% The abstractpage environment sets up everything on the page except
% the text itself.  The title and other header material are put at the
% top of the page, and the supervisors are listed at the bottom.  A
% new page is begun both before and after.  Of course, an abstract may
% be more than one page itself.  If you need more control over the
% format of the page, you can use the abstract environment, which puts
% the word "Abstract" at the beginning and single spaces its text.

%% You can either \input (*not* \include) your abstract file, or you can put
%% the text of the abstract directly between the \begin{abstractpage} and
%% \end{abstractpage} commands.

% First copy: start a new page, and save the page number.
\cleardoublepage
% Uncomment the next line if you do NOT want a page number on your
% abstract and acknowledgments pages.
\pagestyle{empty}
\setcounter{savepage}{\thepage}
\begin{abstractpage}\addcontentsline{toc}{chapter}{Abstract}
In large programming classes, a single problem may yield thousands of student solutions. Solutions can vary in correctness, approach, and readability. Understanding large-scale variation in solutions is a hard but important problem. For teachers, this variation could be a source of innovative new student solutions and instructive examples. Understanding solution variation could help teachers write better feedback, test cases, and evaluation rubrics. Theories of learning, e.g., analogical learning and variation theory, suggest that students would benefit from understanding the variation in the fellow student solutions as well. Even when there are many solutions to a problem, when a student is struggling in a large class, other students may have struggled along a similar solution path, hit the same bugs, and have hints based on that earned expertise.

This thesis describes systems that exploit large-scale solution variation and have been evaluated using data or live deployments in on-campus or edX courses with thousands of students. OverCode visualizes thousands of programming solutions using static and dynamic analysis to cluster similar solutions. Compared to the status quo, OverCode lets teachers quickly develop a high-level view of student understanding and misconceptions and provide feedback that is relevant to more student solutions. Foobaz helps teachers give feedback at scale on a critical aspect of readability, i.e., variable naming. Foobaz displays the distribution of student-chosen names for each common variable in student solutions and, with a few teacher annotations, it generates personalized quizzes that help students learn from the good and bad naming choices of their peers. Finally, this thesis describes two complementary learnersourcing workflows that help students write hints for each other while reflecting on their own bugs and comparing their own solutions with other student solutions. These systems demonstrate how clustering and visualizing solution variation can help teachers directly respond to trends and outliers within student solutions, as well as help students help each other.
\end{abstractpage}

% Additional copy: start a new page, and reset the page number.  This way,
% the second copy of the abstract is not counted as separate pages.
% Uncomment the next 6 lines if you need two copies of the abstract
% page.
% \setcounter{page}{\thesavepage}
% \begin{abstractpage}
% % $Log: abstract.tex,v $
% Revision 1.1  93/05/14  14:56:25  starflt
% Initial revision
%
% Revision 1.1  90/05/04  10:41:01  lwvanels
% Initial revision
%
%
%% The text of your abstract and nothing else (other than comments) goes here.
%% It will be single-spaced and the rest of the text that is supposed to go on
%% the abstract page will be generated by the abstractpage environment.  This
%% file should be \input (not \include 'd) from cover.tex.
Massive programming courses produce a massive collection of student solutions for each programming exercise. The solutions to any particular problem vary along many dimensions, including bugs, naming, syntax, and semantics. The distribution of solutions along these dimensions reflect students' prior knowledge, the teacher's course curriculum and explanations so far, and misconceptions common to all programming students.

Personal tutors can respond to an individual tutee's solution, and only draw on their personal recollection of previously observed solution variation. Teachers teaching larger groups of students can directly observe that a significant fraction of students are struggling with a particular concept or implementation, and respond appropriately with rapid contextual feedback. They can also pick out particular student solutions as examples to illustrate different concepts or ways of solving a problem, rather than solely relying on their own creativity to composing these examples from scratch. 

When scaling up to massive programming courses, it becomes painful or prohibitively exhausting to engage with student solutions this way. It is also an opportunity: teachers have access to a comparatively dense sampling of the distribution over student solution bugs, naming, syntax, and semantics. 

The systems in this thesis help teachers take advantage of the massive collections of solutions, enabling either (1) the same teaching practices that were previously only tractable in smaller courses or (2) new practices that are only possible when a massive collection of student solutions are available. The common empowering mechanism is clustering and visualizing solution variation for human understanding. Using these systems, teachers can gain insights into student design choices, detect autograder failures, process solutions that deserve partial credit, use targeted learnersourcing to collect hints for other students, and give personalized style feedback at scale.

% \end{abstractpage}

\cleardoublepage

\section*{Preface}\addcontentsline{toc}{chapter}{Preface}
I am not a professional software developer, but learning how to write programs in middle school was one of the most empowering skills my father could ever have taught me. I did not need access to chemicals or heavy machinery. I only needed a computer and occasionally an internet connection. I acquired datasets and wrote programs to extract interesting patterns in them. It was and still is a creative outlet.

More recently, the value of learning how to code, or at least how to think computationally, has gained national attention. Last September, New York City Mayor Bill de Blasio announced that all public schools in NYC will be required to offer computer science to all students by 2025. In January of this year, the White House released its Computer Science for All initiative, "offering every student the hands--on computer science and math classes that make them job--ready on day one" (President Obama, 2016 State of the Union Address).

The economic value and employment prospects associated with knowing how to program, the subsiding stigma of being a computer "geek" or "nerd," and the production and popularity of movies about programmers may be responsible for driving up enrollment in computer science classes to unprecedented levels. Hundreds or thousands of students enroll in programming courses at schools like MIT, Stanford, Berkeley and the University of Washington.

One-on-one tutoring is considered a gold standard in education, and programming education is likely no exception. However, most students are not going to receive that kind of personalized instruction. We, as a computer science community, may not be able to offer one-on-one tutoring at a massive scale, but can we create systems that enhance the teacher and student experiences in massive classrooms in ways that would never have been possible in one-on-one tutoring? This thesis is one particular approach to answering that question. %are not even possible in just a one-on-one environment?% How can we best teach the next generation of students how to code when there are so many of them?


\cleardoublepage

\section*{Acknowledgments}\addcontentsline{toc}{chapter}{Acknowledgments}

Thank you, thank you, thank you to my advisor, Rob Miller, who took a chance on me, and my other co-authors on this thesis work, Rishabh Singh, Jeremy Scott, Philip Guo, Lyla Fischer, Aaron Lin, and Carrie Cai. The past and present members of our User Interface Design group were instrumental in helping me feel at home in a new area and get up to speed while having fun. I'm also grateful to my past internship mentors at Google and MSR, specifically Dan Russell, Merrie Ringel Morris, and Andr\'{e}s Monroy-Hern\'{a}ndez, who gave me the chance to try out my chops in new environments. I also want to thank my friends outside MIT, especially the greater Boston community of wrestling coaches, who helped me learn important lessons outside the classroom. And, last but not least, my partner Victor, my parents, and my brother, for all the hugs, proofreading, brainstorming, snack deliveries, technical discussions, and moral support.



%%%%%%%%%%%%%%%%%%%%%%%%%%%%%%%%%%%%%%%%%%%%%%%%%%%%%%%%%%%%%%%%%%%%%%
% -*-latex-*-
